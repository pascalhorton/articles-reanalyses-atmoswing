%% Version 4.3.2, 25 August 2014
%
%%%%%%%%%%%%%%%%%%%%%%%%%%%%%%%%%%%%%%%%%%%%%%%%%%%%%%%%%%%%%%%%%%%%%%
% Template.tex --  LaTeX-based template for submissions to the 
% American Meteorological Society
%
% Template developed by Amy Hendrickson, 2013, TeXnology Inc., 
% amyh@texnology.com, http://www.texnology.com
% following earlier work by Brian Papa, American Meteorological Society
%
% Email questions to latex@ametsoc.org.
%
%%%%%%%%%%%%%%%%%%%%%%%%%%%%%%%%%%%%%%%%%%%%%%%%%%%%%%%%%%%%%%%%%%%%%
%  PREAMBLE
%%%%%%%%%%%%%%%%%%%%%%%%%%%%%%%%%%%%%%%%%%%%%%%%%%%%%%%%%%%%%%%%%%%%%

%% Start with one of the following:
%  DOUBLE-SPACED VERSION FOR SUBMISSION TO THE AMS
\documentclass{ametsoc}

%  TWO-COLUMN JOURNAL PAGE LAYOUT---FOR AUTHOR USE ONLY
%\documentclass[twocol]{ametsoc}

\usepackage{multirow}
\usepackage{gensymb}
% \usepackage[disable]{todonotes} % notes not showed
\usepackage[prependcaption,textsize=small]{todonotes}   % notes showed

%%%%%%%%%%%%%%%%%%%%%%%%%%%%%%%%
%%% To be entered only if twocol option is used

\journal{jcli}

%  Please choose a journal abbreviation to use above from the following list:
% 
%   jamc     (Journal of Applied Meteorology and Climatology)
%   jtech     (Journal of Atmospheric and Oceanic Technology)
%   jhm      (Journal of Hydrometeorology)
%   jpo     (Journal of Physical Oceanography)
%   jas      (Journal of Atmospheric Sciences)	
%   jcli      (Journal of Climate)
%   mwr      (Monthly Weather Review)
%   wcas      (Weather, Climate, and Society)
%   waf       (Weather and Forecasting)
%   bams (Bulletin of the American Meteorological Society)
%   ei    (Earth Interactions)

%%%%%%%%%%%%%%%%%%%%%%%%%%%%%%%%
%Citations should be of the form ``author year''  not ``author, year''
\bibpunct{(}{)}{;}{a}{}{,}

%%%%%%%%%%%%%%%%%%%%%%%%%%%%%%%%

%%% To be entered by author:

%% May use \\ to break lines in title:

\title{Impact of the global reanalysis dataset on statistical precipitation downscaling.}

%Comparison of global reanalysis datasets in the context of statistical precipitation downscaling.

%%% Enter authors' names, as you see in this example:
%%% Use \correspondingauthor{} and \thanks{Current Affiliation:...}
%%% immediately following the appropriate author.
%%%
%%% Note that the \correspondingauthor{} command is NECESSARY.
%%% The \thanks{} commands are OPTIONAL.

    %\authors{Author One\correspondingauthor{Author One, 
    % American Meteorological Society, 
    % 45 Beacon St., Boston, MA 02108.}
% and Author Two\thanks{Current affiliation: American Meteorological Society, 
    % 45 Beacon St., Boston, MA 02108.}}

\authors{Pascal Horton\correspondingauthor{University of Bern, Institute of Geography, Hallerstrasse 12, 3012 Bern, Switzerland.}, Rolf Weingartner, Stefan Br\"{o}nnimann}

%% Follow this form:
    % \affiliation{American Meteorological Society, 
    % Boston, Massachusetts.}

\affiliation{Oeschger Centre for Climate Change Research, Institute of Geography, University of Bern, Bern, Switzerland}

%% Follow this form:
    %\email{latex@ametsoc.org}

\email{pascal.horton@giub.unibe.ch}

%% If appropriate, add additional authors, different affiliations:
    %\extraauthor{Extra Author}
    %\extraaffil{Affiliation, City, State/Province, Country}

\extraauthor{Charles Obled}
\extraaffil{Institut des G\'{e}osciences de l'Environnement, Universit\'{e} de Grenoble-Alpes, Grenoble, France}


%%%%%%%%%%%%%%%%%%%%%%%%%%%%%%%%%%%%%%%%%%%%%%%%%%%%%%%%%%%%%%%%%%%%%
%  ABSTRACT
%
% Enter your abstract here
% Abstracts should not exceed 250 words in length!
%
% For BAMS authors only: If your article requires a Capsule Summary, please place the capsule text at the end of your abstract
% and identify it as the capsule. Example: This is the end of the abstract. (Capsule Summary) This is the capsule summary. 

\abstract{
	Statistical downscaling based on a perfect prognosis approach often rely on reanalyses to infer the statistical relationship between synoptic predictors and the local variable of interest, here daily precipitation. Nowadays, many reanalyses are available and their impact on the downscaled variable is not often considered. The present work assessed the impact of ten reanalyses on the performance of seven analogue methods for statistical precipitation downscaling at 300 stations in Switzerland. Despite being in a data-rich region, significant differences were found between reanalyses and their impact on the performance of the method can even be higher than the choice of the predictor variables. There were no single overall winner, but a selection of datasets resulting in higher skills. 
	The methods based on geopotential heights were not highly sensitive to the spatial resolution of the output products. As expected, more local variables, such as moisture, were more sensitive to the spatial resolution. 
	Reanalyses with longer archives allow increasing the pool of potential analogues resulting in higher performances. However, when adding variables affected by errors in a more distant past, the skill decreased again.
	The use of multiple members from two reanalyses was also tested over a recent and a past period. The interest of using members to increase the performance by better enfolding the uncertainties was found to be limited, and even problematic with methods using multiple analogy levels.
}


\begin{document}

%% Necessary!
\maketitle

%TODO: Update figs 2, 3, 4, 5, 7, 8, 10, 11, 12 with last version
%TODO: Publish analogue dates + params as datasets (not values & scores) ?



%https://climatedataguide.ucar.edu/climate-data/era-20c-ecmwfs-atmospheric-reanalysis-20th-century-and-comparisons-noaas-20cr
%https://www.esrl.noaa.gov/psd/data/gridded/data.ncep.reanalysis.derived.html
%https://www.esrl.noaa.gov/psd/data/20thC_Rean/
%https://climatedataguide.ucar.edu/climate-data/atmospheric-reanalysis-overview-comparison-tables


\todo{title}
\todo{try monthly / seasonal variability}
\todo{explain spatial domains over EU for predictors}
\todo{find explanations in other papers}


%%%%%%%%%%%%%%%%%%%%%%%%%%%%%%%%%%%%%%%%%%%%%%%%%%%%%%%%%%%%%%%%%%%%%
%  MAIN BODY OF PAPER
%%%%%%%%%%%%%%%%%%%%%%%%%%%%%%%%%%%%%%%%%%%%%%%%%%%%%%%%%%%%%%%%%%%%%
%

%% In all cases, if there is only one entry of this type within
%% the higher level heading, use the star form: 
%%
% \section{Section title}
% \subsection*{subsection}
% text...
% \section{Section title}

%vs

% \section{Section title}
% \subsection{subsection one}
% text...
% \subsection{subsection two}
% \section{Section title}

%%%
% \section{First primary heading}

% \subsection{First secondary heading}

% \subsubsection{First tertiary heading}

% \paragraph{First quaternary heading}


\section{Introduction}

Statistical downscaling is widely used to bridge the resolution gap between climate model outputs and impact models, and to bias-correct them. Several of these methods consist in establishing an empirical statistical relationship between large-scale atmospheric variables and the local variable of interest. Following the classification of \citet{Rummukainen1997} also used in \citet{Maraun2010}, there are basically two types of approaches: perfect prognosis, for which the relationship is calibrated on large-scale and local-scale observations, and model output statistics, for which the relationship is calibrated on model outputs for the large-scale variables and local-scale observations. In the context of perfect prognosis approaches, large-scale observations are thus needed. Global meteorological reanalyses are very useful to fulfill this role, as they provide gridded large-scale variables available at any location of the world.

Reanalyses are produced with a single version of a data assimilation system with a forecast model constrained to follow at best the observations over a long period. They provide multivariate outputs that are physically consistent, and that contain information in locations where few or no observations are available, also for variables that are not directly observed \citep{Gelaro2017}. Their accuracy depends on both the quality of the model physics and that of the analysis process, and thus indirectly of the quantity and quality of the assimilated observations \citep{Dee2011a}. The homogeneity of a reanalysis is a challenge due to significant changes in observing systems. For example, the availability of satellite observations drastically changed the amount of data available, particularly for regions with sparse conventional observation networks. The assimilation of a variable amount of observations is likely to lead to inhomogeneities in the reanalysis. For this reason, some reanalyses are either limited to the satellite era, or, if the period of interest expands before 1980, they might not use satellite observations. Because of these discontinuities in the available observations, some variables, such as precipitation and evaporation are to be used with great caution \citep{Kobayashi2015}.

In perfect prognosis methods, reanalyses are considered as observations of the large-scale variables, and are supposed perfect. As the statistical relationship established using reanalyses is afterwards applied to outputs from other models, the variables considered as predictors should not strongly depend on the forecast model used and should be as robust as possible, so that they are mainly influenced by the observations only. Moreover, when looking for analogue days in a reanalysis to a situation defined by another forecast model, it is preferable to consider products that are as similar as possible.

The present work focuses on the analogue method (AM), which is a statistical downscaling technique relying on the hypothesis that similar synoptic situations are likely to result in similar local effects, plus a certain variability that is not explained by the considered predictors \citep{Lorenz1969}. The variable of interest is here precipitation. In order to take into account the unexplained variability, several analogue days are selected and their observed precipitation values are used to provide the empirical conditional distribution that is the statistical prediction for the considered target day. Different versions of AMs exist, relying on various predictors. However, they generally contain a predictor describing the atmospheric circulation.

In one of the first version of the AM, the predictors were extracted from radio-sounding data \citep{Duband1981}, which involved heavy pre-treatment to have a complete and homogeneous database that could be used. Other authors worked with rather short local analysis from forecast models \cite[for example][]{Kruizinga1983, VandenDool1989}. The release of the first reanalysis \citep[NCEP/NCAR Reanalysis I, NR-1,][]{Kalnay1996, Kistler2001} greatly simplified the implementation of the AM, and opened new opportunities of using other variables that were not available before. This increased the popularity of the method \citep{Timbal2008a}.

\citet{Timbal2003} and \citet{Bontron2004} were the first authors to use NR-1 in the AM. As NR-1 and its updated version NCEP/DOE Reanalysis 2 \citep[NR-2,][]{Kanamitsu2002} remained popular for a long time, they were often used until recently in AMs \citep{Wetterhall2005a, Gangopadhyay2005, Altava-Ortiz2006, Barrera2007, Cannon2007, Matulla2007, Bliefernicht2007, Maurer2008, Wu2012, Marty2012, Teng2012, Horton2012, Yiou2014}. When the first European long reanalysis ERA-40 \citep{Uppala2005} became available, it also became popular for European users, but after a delay of some years \citep {Willems2011b, JakobThemessl2011a, BenDaoud2011, Turco2011a, Franke2011, Pascual2012b, Schenk2012, Ribalaygua2013a, Osca2013, Radanovics2013, Martin2014b, Chardon2014, BenDaoud2016}. \citet{BenDaoud2009} analyzed the impact of choosing NR-1 or ERA-40 in the AM developed by \citet{Bontron2004} and found no significant difference for the predictors considered. The more recent ERA-Interim \citep[ERA-INT, ][]{Dee2011a} was used by \cite{Raynaud2016b}, and MERRA \citep{Rienecker2011} was used by \citet{Vanvyve2015}. Several recent reanalysis products are not yet used in AMs.

In almost all of these works, a single reanalysis was used. The choice is likely to be primarily driven by the ease of access and the availability of some datasets in the research units, along with the code able to read it. Indeed, it is often not a priority to use the last reanalysis available if the benefit for AMs is not proven, as it requires efforts to acquire ever larger datasets and to adapt the code to read it. Moreover, they are often considered as equivalent for a data-rich region such as Europe. It likely explains why many authors stick with the first American or European reanalyses.

Some other use of the AM focus on the reconstruction of weather conditions in the Twentieth Century. Then, they require reanalyses spanning this period, such as the ECMWF twentieth century reanalysis \citep[ERA-20C, ][]{Poli2016} or the Twentieth Century Reanalysis \citep[20CR][]{Compo2011} produced by NOAA \citep[for example,][]{Kuentz2015, Caillouet2016, Brigode2016, Bonnet2017}. 

To our knowledge, \citet{Dayon2015} made the most comprehensive comparison of the reanalyses in the AM so far. They compared NR-1, MERRA, ERA-INT and 20CR and noted that the choice of the reanalysis, even if not crucial for climate change impacts, is a non-negligible source of uncertainty, and that it can even impact the performance of the method to a greater extent than the choice of the predictors. They concluded that "the substantial differences in downscaling results associated with reanalyses [...] suggests that the role of reanalyses should not be underestimated when evaluating the statistical downscaling method". The choice of the predictors was also found to vary from a reanalysis to another, in a way that the optimization of the method is likely to be reanalysis dependent and that using a single reanalysis might introduce a lack of robustness \citep{Dayon2015}.

Reanalyses were compared in multiple studies (... \todo{Add some references}), but mainly in terms of climatological properties. It is certainly important for AMs to use reanalyses with accurate climatological characteristics, and particularly a good homogeneity over the period covered. However, AMs are sensitive to daily variability, and situations related to heavy precipitation events have a significant influence on the performance assessment. The impact of the reanalysis on the performance of the AM can not be directly deduced from climatological analyses and needs to be assessed within the method, which is the goal of the present work. Ten reanalyses were compared for seven AMs at 300 stations in Switzerland (Sect. \ref{sec:influence}). Additionally, the role of the spatial resolution (Sect. \ref{sec:analyzes}.\ref{sec:resolution}), the length of the archive (Sect. \ref{sec:analyzes}.\ref{sec:length}), and the use of different members from ensemble datasets (Sect. \ref{sec:analyzes}.\ref{sec:ensemble}) were investigated.


\section{Data and methods}
\label{sec:data}

\subsection{Reanalysis datasets}

The considered global atmospheric reanalyses are briefly described hereafter, ordered chronologically, and some of their characteristics are provided in Table \ref{table:datasets}. The common period to all datasets is 1981--2010. All reanalyses are available at a 6-h time step, with some products or variables having higher temporal resolution. Only the 6-h time step is considered in the present work. All reanalyses are based on observations selected after quality control, which will not be detailed here. 


\subsubsection{NCEP Reanalysis I}

The NCEP/NCAR Reanalysis I \citep[NR-1,][]{Kalnay1996, Kistler2001} was the first global reanalysis product. It is made with a forecast model frozen at the state-of-the-art of 1995 and by assimilating land surface, ship, aircraft, rawinsonde, and satellite data among others. However, upper-air observations have a much larger influence on the analysis than the surface observations \citep{Kistler2001}. The data assimilation system is a 3D variational technique (3D-Var). The model resolution is T62 (about 210~km) with 28 sigma vertical levels. All major physical processes are parameterized. The period coverage was first starting in 1957, before being extended back to 1948. \citet{Kalnay1996} were aware that assimilating all available data at a given time would have an impact on the climate of the reanalysis due to changes in the observating system, but the choice was made for accuracy over stability of the climate. A comparison of two sets of analyses made with and without the use of satellite data showed that even without satellite data, almost 100\% of the daily variance of the geopotential height was explained in the Northern Hemisphere (NH) extratropics \citep{Kalnay1996}. Lower correlation values were found in other regions of the globe, particularly in the Southern Hemisphere (SH), where the uncertainty is much higher due to the lack of rawinsonde data. However, RMS of the analysis increments (difference between the forecast and the analysis) at 500~hPa showed large differences between a data-poor year (1958) and a data-rich year (1996), and the climatologies before and after 1979 differ significantly due to the use of satellite data \citep{Kistler2001}.


\subsubsection{NCEP Reanalysis II}

The NCEP/DOE Reanalysis 2 \citep[NR-2,][]{Kanamitsu2002} is a follow-on to the NCEP/NCAR Reanalysis I project with the goal to correct known problems that were identified after the NR-1 production, mainly human processing errors. However, these issues have consequences for a limited number of applications. NR-2 also relies on updated versions of the assimilation system and the forecast model, with improvements to the model physics. Changes in parameterizations have improved the precipitation estimate, but may have caused deterioration of other variables \citep{Kistler2001, Kanamitsu2002}. Primary analysis variables, such as geopotential heights, only contains minor differences with NR-1. The model and the outputs have the same spatial and temporal resolution as NR-1, and mostly the same observational data were assimilated.


\subsubsection{ERA-Interim}

ERA-Interim \citep[ERA-INT, ][]{Dee2011a} is produced by the European Centre for Medium-Range Weather Forecasts (ECMWF) and covers the period from 1979 onwards. It replaced ERA-40 \citep{Uppala2005}, which replaced ERA-15 \citep{Gibson1997}, reanalyses of 45 and 15 years respectively. ERA-INT aims at addressing problems in data assimilation of ERA-40 and at improving several technical aspects in the process, which are expected to have an impact on the quality of the product.

ERA-INT uses a 4D variational technique (4D-Var) with sequential data assimilation in 12-hourly analysis cycles. It assimilates conventional observations, rawinsonde, aircraft, and satellite data, etc. 4D-Var is expected to make a more effective use of observations \citep{Dee2011a}. ERA-INT also relies on several bias and error correction techniques that were introduced after ERA-40 in order to minimise inconsistencies between observations of different types.

The forecast model uses a hybrid sigma-pressure vertical coordinate on 60 layers and has a T255 horizontal resolution (about 79~km) and a 30~min time step. Orographic effects and convection schemes, among others, have been improved since ERA-40.


\subsubsection{Climate Forecast System Reanalysis}

The Climate Forecast System Reanalysis \citep[CFSR, ][]{Saha2010a} is the second generation of global reanalyses provided by NCEP. The model resolution was significantly increased since NR-1/NR-2 : horizontal resolution of T382 (about 38~km) and 64 vertical levels on sigma-pressure hybrid vertical coordinates. Both the forecast model and the assimilation were improved, and a coupling to the ocean as well as a sea-ice model were introduced. New parameterizations were used, resulting in more realistic moisture prediction and mountain blocking representation, among others \citep{Saha2010a}. Temperature, moisture, and ozone fields are also better adjusted to best match the observed radiances.

CFSR has the particularity to ingest the historical tropical storm locations. The substantial benefit is that "by relocating a tropical storm vortex to its observed location prior to the assimilation of storm circulation observations, distortion of the circulation by the mismatch of guess and observed locations is avoided" \citep{Saha2010a}. This aspect can have significant impact on geopotential heights in certain regions of the world.

The assimilation scheme relies on the 3D-Var technique, but with a certain consideration of the time aspect by using time tendencies of state variables. The analysis system used in CFSR for the atmosphere is similar to the one used by MERRA, with nearly the same input data. The assimilated data are most of available observations from different kind of sensors. The period covered is from 1979 onwards, but with the plan to extend it back to 1947 or earlier \citep{Saha2010a}.


\subsubsection{Japanese 55-year Reanalysis}

The Japanese 55-year Reanalysis \citep[JRA-55, ][]{Kobayashi2015, Harada2016} is the second global atmospheric reanalysis produced by the Japan Meteorological Agency (JMA). It starts in 1958, which makes it the first reanalysis applying 4D-Var to this period. The forecast model used has a TL319 spectral resolution (about 60~km) and 60 vertical layers. The improvement in the forecast model and the assimilation, as well as the use of newly available and improved observations have led to substantial improvements in the reanalysis quality since JRA-25 \citep{Kobayashi2015, Harada2016}, the first Japanese product. The observations used consist of those archived by JMA and those used in ERA-40 \citep{Uppala2005}. Tropical cyclones data are also assimilated, which results in a good representation of tropical cyclones in the dataset compared to other reanalyses \citep{Harada2016}. JRA-55 is still sensitive to changes in the observing networks for some characteristics, but far less than JRA-25 was, which is likely related to improvements in the forecast model providing greater physical consistency of the JRA-55 product \citep{Kobayashi2015}. JRA-55 aims at providing time-consistent data suitable for climate change studies \citep{Ebita2011}.

JMA also released another product, JRA-55 Conventional \citep[JRA-55C,][]{Kobayashi2014}, a version of the reanalysis based on the assimilation of only conventional data and upper air observations, without any satellite observation. The dataset is thus more homogeneous as it is unaffected by changes in satellite observing systems, even though the time variation of the number of observations may also have an impact. The data assimilation system and the boundary conditions remains identical to the ones used in JRA-55. JRA-55C starts in 1972; the full 55-years reanalysis is obtained by using outputs from JRA-55 prior to 1972. Geopotential heights at 500~hPa are found to present very small differences in the extra-tropics in the Northern Hemisphere. These differences are more important for the Southern Hemisphere. Globally, the anomaly of geopotential heights are highly correlated between both datasets, except where conventional observations are sparse, especially for high latitude areas of the SH. Precipitation, which provides an integrated evaluation of the performances of the assimilation system, is more consistent with observations in some analyzed regions for JRA-55 and JRA-55C than for JRA-25 and ERA-INT.


\subsubsection{NOAA-CIRES 20th Century Reanalysis}

The Twentieth Century Reanalysis version 2c \citep[20CR-2c, ][]{Compo2011} produced by NOAA starts in 1851. Unlike the other reanalyses that assimilate as much data as possible, it only assimilates surface pressure data and relies on observed monthly sea-surface temperature and sea-ice distributions as boundary conditions. The assimilation technique used is an Ensemble Kalman Filter (EnKF) that allows to take into account the time-varying observations uncertainty related to the evolution of the measuring networks. The atmospheric model used is the NCEP Global Forecast System (GFS) with a T62 horizontal resolution and 28 vertical hybrid sigma-pressure levels. The reanalysis contains 56 members and an ensemble mean. As expected, the ensemble uncertainty varies with the time-changing observation network, i.e. it decreases over time. The outputs are available with a 2\degree\ resolution on 24 pressure levels (for the ensemble mean -- less levels are publicly available for the individual members).

Although 20CR-2c only relies on surface data, it has relevant information of the state of the atmosphere at higher levels, such as the 500~hPa geopotential height and the 850~hPa air temperature \citep{Compo2011}. Statistics of storm tracks and some climate indices are also relevant \citep{Compo2011}.


\subsubsection{ECMWF 20th Century Reanalysis}

The ECMWF twentieth century reanalysis \citep[ERA-20C, ][]{Poli2016} starts in 1900. Unlike 20CR-2c, it is single-member. ERA-20C also assimilates surface pressure, but additionally assimilates marine wind observations. It is forced by sea surface temperature, sea ice cover, atmospheric composition changes, and solar forcing. The forecast model used is the ECMWF’s Integrated Forecast System (IFS) with a time step of 30 min, a T159 resolution (approximately 125 km), and 91 levels. The assimilation technique is 4D-Var on a 24~h window, which is also able to account for spatially and temporally varying errors in the model and the observations, as in 20CR-2c. A previously produced 10-member ensemble was used to derive these errors estimates.

When analysing the total column water vapor output, \citet{Poli2016} found that ERA-20C and 20CR-2c have dry biases but better reproduce anomalies than JRA-55 or ERA-INT. Despite some differences, climate indices are in reasonable agreement between reanalyses.


\subsubsection{MERRA-2}

The Modern-Era Retrospective Analysis for Research and Applications, version 2  \citep[MERRA-2, ][]{Gelaro2017} is an improvement of the first MERRA reanalysis \citep{Rienecker2011} produced by NASA's Global Modeling and Assimilation Office (GMAO). One of its the main objective is to improve the hydrological cycle represented in reanalysis products, primarily by providing improvement in precipitation and water vapor climatology.

MERRA-2 relies on the assimilation of additional satellite observations than MERRA, including deduced surface wind speeds and vectors. The quality control procedure was changed in order to allow for larger observation values (outliers) to be preserved, for surface pressure and other variables. An important improvement is that it shows a reduction of biases and imbalances in the water cycle, and a reduction of discontinuities in precipitation related to changes in the observing system \citep{Gelaro2017}.

The forecast model has also improved both in its dynamical core and its physical parameterizations since the first version of MERRA. A particularity of MERRA-2 compared to the other reanalyses considered in the present work is that it uses a finite-volume dynamical core with a cubed-sphere horizontal discretization rather than a spectral model. The model grid has a relatively uniform resolution of 0.5\degree\ x 0.625\degree\ with 72 vertical levels.


\subsubsection{ECMWF Coupled 20th Century Reanalysis}

The ECMWF coupled twentieth century reanalysis (CERA-20C) is a follow-up of ERA-20C, with an additional coupling to the ocean and a more recent version fo the IFS model (P. Laloyaux, pers. comm., November 15, 2017). It is an ensemble dataset containing 10 members and spanning the period 1901--2010. The additional assimilated data are ocean temperature and salinity profiles. The coupled data assimilation system is able to take into account feedbacks between the ocean and atmosphere in the forecast as well as the analysis step \citep{Laloyaux2016}. This ensures a physical consistency between the upper ocean and the lower atmosphere. The positive impact of the coupling on temperatures is limited for the atmosphere, but more important for the ocean \citep{Laloyaux2016}. Changes in the atmosphere temperature are located near the ocean surface, but there is no impact for the upper atmosphere. The coupled system has shown a neutral impact for geopotential heights or wind speeds \citep{Laloyaux2016}.



\subsection{Precipitation dataset}
\label{sec:precip}

The predictands -- variables to be predicted -- considered here were daily precipitation totals (06:00 h UTC to 06:00 h UTC the following day) at 301 weather stations of the MeteoSwiss network in Switzerland (Fig. \ref{fig:stations}). All stations with a good data record over the period 1981--2010 were considered. Often, applications of analogue methods rely on gridded precipitation or catchment-scale aggregated series, but any data manipulation were avoided here in order to obviate any undesired interference with the sensitivity analysis. The precipitation data were not transformed by a square root like in some other studies \cite[see e.g.][]{Bontron2004}. Thirty stations -- those with longer time series -- were selected for additional analyses (Sect. \ref{sec:analyzes}).

The 30-year precipitation dataset was divided into a calibration period (CP) and an independent validation period (VP). In order to reduce the impact of potential inhomogeneities in the time series, the selection of the VP was evenly distributed over the entire series \citep[as in][]{BenDaoud2010}. A total of 6 years was considered for the VP by selecting 1 out of every 5 years. Days from the VP were never used as candidate situations for the selection of analogues.


\subsection{Considered analogue methods}
\label{sec:ams}

Different variants of the AM were considered in the present work (Table \ref{table:methods}). These methods have varying degrees of complexity and are constituted of either a single or multiple subsequent levels of analogy with predictor variables of different kind. A relatively simple method that is often considered as reference is based on the analogy of synoptic circulation on two geopotential heights (Z1000 at 12:00 h UTC and Z500 at 24 h UTC) and is named here 2Z.

The 2Z method consists of the following steps: first, to cope with seasonal effects, candidate dates are extracted within a period of four months centred around the target date, for every year of the archive (PC: preselection on calendar basis in Table \ref{table:methods}). Then, the similarity of the atmospheric circulation of a target date with every day from the preselection set (excluding the same year as the target day) is assessed by processing the S1 criterion \citep[Eq.\ \ref{eq:S1}, ][]{Teweles1954, Drosdowsky2003}, which is a comparison of gradients, over a certain spatial window (domain on which the predictors are compared):

\begin{equation}
\label{eq:S1}
S1=100 \frac {\displaystyle \sum_{i} \vert \Delta\hat{z}_{i} - \Delta z_{i} \vert}
{\displaystyle \sum_{i} max\left\lbrace \vert \Delta\hat{z}_{i} \vert , \vert \Delta z_{i} \vert \right\rbrace }
\end{equation}
where $\Delta \hat{z}_{i}$ is the difference in geopotential height between the \textit{i}-th pair of adjacent points of gridded data describing the target situation, and $\Delta z_{i}$ is the corresponding observed geopotential height difference in the candidate situation. The smaller the S1 values, the more similar the pressure fields. This criteria being processed on gradients, it is not sensitive to biases in the considered predictors, as long as the circulation is correctly represented.

The $N_{1}$ dates with the lowest values of S1 are considered as analogues to the target day. The number of analogues, $N_{1}$, is a parameter to calibrate. Then, the daily observed precipitation values for the $N_{1}$ selected dates provide the empirical conditional distribution, considered as the probabilistic prediction for the target day. The choice of the predictors for 2Z, and their corresponding pressure levels and temporal windows were optimized for NR-1 \citep{Bontron2004}.

A variation of the former method, but based on the mean sea level pressure (2SLP) rather than geopotential heights was also assessed in this work. The S1 criteria was also used to quantify the analogy between the pressure fields. SLP was used in AMs by \citet{Zorita1999}, \citet{Timbal2001a} and \citet{Martin2014b}, amongst others.

Another method relying only on the atmospheric circulation has also been considered. It relies on four geopotential heights (4Z, Table \ref{table:methods}) that were automatically selected by genetic algorithms for the upper Rhone catchment in Switzerland \citep{Horton2017b}. The 4Z method was shown to outperform 2Z by exploiting more information from the geopotential heights and by taking advantage of additional degrees of freedom, such as different spatial windows between the pressure levels and the introduction of a weighting between them. However, due to the high number of reanalyses and stations considered in this work, it was not possible to use genetic algorithms in order to optimize the method. Thus, the 4Z method considered here is a simplification of the results from \citet{Horton2017b}, and only the selection of the optimal pressure levels and temporal windows were considered (Z1000 at 06:00 and 30:00 h UTC, Z700 at 24:00 h UTC, and Z500 at 12:00 h UTC), and used for all stations. As for the 2Z method, a unique but station-specific spatial window for all pressure levels was considered, and the weights between the pressure levels had equal values. Such simplifications of the parameters resulted in a decrease of the performance score, which, however, was still superior that of 2Z.

The other methods considered hereafter add a second, or more, subsequent level(s) of analogy after the analogy of the atmospheric circulation. Unlike stated in \citet{Caillouet2016}, stepwise analogue methods existed for some time \citep[e.g.][]{Bontron2004, Bontron2005, Marty2010, Marty2012, Horton2012a}. 

The next parametrization adds a second level of analogy on the moisture variables (method 2Z-2MI, Table \ref{table:methods}). The predictor that \citet{Bontron2004} found optimal for France is a moisture index (MI) made of the product of the total precipitable water (TPW) with the relative humidity at 850~hPa (RH850). \cite{Horton2012a} confirmed that this index is also better for the Swiss Alps than any other variable from NR-1 considered independently. When adding a second level of analogy, $N_{2}$ dates are subsampled from the $N_{1}$ analogues of the atmospheric circulation, to end up with a smaller number of analogue situations. When this second level of analogy is added, a higher number of analogues $N_{1}$ is kept on the first level. 

Similarly to the 4Z method, 4Z-2MI is a simplification of the methods optimized by genetic algorithms in \citet{Horton2017b}. It consists of a first level of analogy on four geopotential heights (Z1000 at 30:00 h UTC, Z850 at 12:00 h UTC, Z700 at 24:00 h UTC, and Z400 at 12:00 h UTC) followed by the moisture index (MI) at two pressure levels (MI700 at 24:00 h UTC and MI600 at 12:00 h UTC).

\citet{BenDaoud2016} replaced the calendar preselection ($\pm$ 60 days around the target date) by a preselection on similar air temperature (T925 at 36:00 h UTC and T600 at 12:00 h UTC, at the nearest grid point). It allows a more dynamic screening of similar situations in terms of air masses as the seasonal signal is also present in the temperature data. The undesired mixing of spring and autumn situations is discussed in \citet{Caillouet2016}. The number of preselected dates ($N_{0}$) is equivalent to the number of days one would have chosen with the calendar approach, and thus depends on the archive size: $N_{0} = 120 \cdot n_{a}$ where $n_{a}$ is the number of years in the calibration period. In this method, named PT-2Z-4MI, the analogy of the atmospheric circulation is the same as in the 2Z method, but the moisture analogy differs (MI925 and MI700 at 12:00 and 24:00 h UTC).

Subsequently, \citet{BenDaoud2016} added an additional level of analogy between the circulation and the moisture analogy (PT-2Z-4W-4MI, Table \ref{table:methods}) based on vertical velocity at 850~hPa (W850). This AM was primarily developed for large floodplains in France (Sa\^{o}ne, Seine) and is the most complex method considered in this work. 

Precipitation variables from reanalyses are generally not considered as predictors, as they strongly depend on the model physics \citep{Rienecker2011} and are characterized by important biases, which would make the method non-transposable to the outputs of another model. \citet{Dayon2015} assessed the relevance of using precipitation from four reanalyses as predictors. This resulted in downscaled precipitation with high inter-annual correlation with the observations for MERRA, but the related biases were much larger than for other predictors. Their analyses conducted them to reject precipitation as a predictor.


\subsection{Calibration of the AMs}
\label{sec:calibration}

AMs rely on parameters that need to be defined for every level of analogy. In this work, the choice of the predictors and their corresponding temporal windows (hour of the day) was identical to the listed methods in Table \ref{table:methods} and were not reassessed. The parameters that were here calibrated for every station, method, and reanalysis, are:

\begin{itemize}
	\item The spatial windows, which are the domains on which the predictors are compared. A spatial window is specific to each level of analogy and is here shared between all the predictors of that level.
	\item The optimal number of analogues to sample for every level of analogy.
\end{itemize}

The semi-automatic sequential procedure developed by \citet{Bontron2004} was used to calibrate the AM. The procedure used is described in \citet{Horton2017c} and is similar to the work of \citet{Radanovics2013} and \citet{BenDaoud2016}. It was implemented in the open source AtmoSwing-optimizer software v1.5.0 \citep[www.atmoswing.org,][]{Horton2017a}, which was used to perform the calibration.

When calibrating the method, it is achieved to maximize or minimize an objective function. The CRPS \citep[Continuous Ranked Probability Score,][]{Brown1974, Matheson1976, Hersbach2000} is often used for that purpose. It allows evaluating the predicted cumulative distribution functions $F(y)$, for example of the precipitation values $y$ associated with the analogue situations, compared to the single observed value $y^{0}$:

\begin{equation}
\label{eq:CRPS}
CRPS = \int_{-\infty}^{+\infty} \left[ F_{i}(y)-H_{i}(y-y_{i}^{0})\right]^{2} dy
\end{equation}
where $H(y-y_{i}^{0})$ is the Heaviside function that is null when $y-y_{i}^{0}<0$, and has the value 1 otherwise. The better the prediction, the smaller the score.

In order to compare the value of the score relative to a reference, one often considers its skill score expression, and uses the climatological distribution of precipitation from the entire archive as the reference. However, the choice of the reference is not important when comparing performances. The CRPSS (Continuous Ranked Probability Skill Score) is thus defined as follows \citep{Bradley2011}:

\begin{equation}
\label{eq:CRPSS}
CRPSS = 1-\frac{\overline{CRPS}}{\overline{CRPS}_{clim}}
\end{equation}
where $CRPS_{clim}$ is the CRPS value for the climatology. A better prediction is characterized by an increase in CRPSS.


\section{Impact of the reanalysis}
\label{sec:influence}

All considered AMs were calibrated for every reanalysis and station, which resulted in a total of 21,070 calibrations processed on a HPC cluster at the University of Bern. For every combination, the spatial windows and the number of analogues of each analogy level were calibrated (Sect. \ref{sec:data}\ref{sec:calibration}) to always be optimal. Unless stated otherwise, all results are presented for the VP (independent validation period, Sect. \ref{sec:data}\ref{sec:precip}). Results on the CP (calibration period) were similar. The comparison period was 1981--2010. The original spatial resolution of the reanalyses were used and thus differ from one to another. The impact of the spatial resolution is analysed in Sect. \ref{sec:analyzes}.\ref{sec:resolution}.

The results of 20CR-2c are shown for the ensemble mean only. The same analyses were performed on a single member (the first one), but no significant difference has been observed. The single-member was slightly less skilful than the ensemble mean, but to a negligible extent (not shown).

One has to keep in mind that biases in the variables might not affect the performance of the AM, as long as they are constant over time and the methods are used in a perfect prognosis framework. For example, a constant bias in the values of Z will not alter the selection of analogues, whereas a bias in the circulation frequency will affect the performance.


\subsection{Impact on the skill}

The CRPSS scores of all considered AMs and reanalyses are shown in Fig. \ref{fig:comparison_values}. As \citet{Dayon2015} also observed, the reanalysis had an impact on the skill of the AM that can be even higher than the choice of predictors, and is thus a non negligible source of uncertainty. Globally, the skill tends to increase with the complexity of the AM. The two first methods based on two circulation predictors, 2SLP and 2Z, were equivalent, except for MERRA-2, where SLP showed a higher predictive skill than Z. Then, there was a systematic increase of the skill from 2Z, 4Z, 2Z-2MI, up to 4Z-2MI. Finally, the respective performance of 4Z-2MI, PT-2Z-4MI and PT-2Z-4W-4MI varied from one reanalysis to another. The spread was relatively similar between reanalyses.

The effect of the reanalysis was isolated in Fig. \ref{fig:comparison_relative} by processing the difference in CRPSS for one reanalysis compared to the mean performance on all reanalyses, per station and per method. The variability was reduced because the climatological differences between the stations were mostly removed. Except for 2SLP, there is a tendency for the impact of the reanalysis to increase with the complexity of the method. This is particularly visible for ERA-INT, JRA-55, JRA-55C and 20CR-2c. The spread cannot be interpreted in Fig. \ref{fig:comparison_relative}, as it is more akin to the average performance on all reanalyses.

In general, modern reanalyses that assimilate upper air observations performed better for this region of the globe, independently of the assimilation technique or the availability of high resolution outputs. It includes ERA-INT, CFSR, JRA-55, JRA-55C, and MERRA-2. Older reanalyses, namely NR-1 and NR-2, and those not assimilating upper air observations, such as 20CR-2c, ERA-20C and CERA-20C showed lower skills.

The two first reanalyses NR-1 and NR-2 were mostly slightly below the average. ERA-INT performed generally well, except for 2SLP and 2Z, where, surprisingly, it did not surpass NR-1 and NR-2. The addition of more geopotential heights or moisture variables brought it to the top selection (from 4Z on). CFSR was always in the best reanalyses, except when vertical velocity in used, which decreased slightly its performance. The two Japanese reanalyses JRA-55 and JRA-55C also performed well. In the top selection, JRA-55C is the only reanalysis that does not assimilate satellite observations, but it showed here the same performance that JRA-55. 20CR-2c systematically resulted in lower performances, and its relative skill significantly decreased for more complex methods, even though the absolute skill score generally increased (Fig. \ref{fig:comparison_values}). It means that moisture, temperature and vertical velocity from 20CR-2c were not as informative for the AM than those from other reanalyses. ERA-20C, which only assimilates surface pressure and wind observations, had an impact situated around the average. It did perform slightly better than NR-1 and NR-2, and largely better than 20CR-2c, but not as good as the reanalyses assimilating upper air observations. MERRA-2 was part of the top selection, and its SLP was found to be particularly skilful compared to other reanalyses. 2SLP with MERRA-2 was found to perform even better than using four geopotential heights. CERA-20C performed similarly to ERA-20C, but was a bit better for 2SLP, likely due to the coupling to the ocean.

The impact of the reanalysis was then investigated by precipitation thresholds for the target day (not shown). For dry days, there were large differences between reanalyses. Globally, the same reanalyses performed better (ERA-INT, CFSR, JRA-55, JRA-55C, and MERRA-2), and the method PT-2Z-4W-4MI was a bit superior. For days with high precipitation, 4Z-2MI was slightly superior, with differences between reanalyses that were not so big, to the exception of 20CR-2c with an ever larger negative difference with the growing complexity of the method. The relative overall lower performance of 20CR-2c was related to both precipitation events and dry days when using moisture variables. The prediction of dry days with circulation-only variables from 20CR-2c presented the same skills as NR-1, NR-2, ERA-20C, and CERA-20C. MERRA-2 had in average the largest positive influence to predict dry days, although its remarkable skill for 2SLP seems to be first related to days with rain. The lower performance of ERA-INT for 2SLP and 2Z came from days with high precipitation, whereas JRA-55 and JRA-55C were in average slightly superior to the others for these situations.

Daily correlations were processed between the median or the mean precipitation from the selected analogues and the observations. The results were similar to Fig. \ref{fig:comparison_values} (but with higher values) and are thus not presented. The inter-annual correlation was processed in the same way (Figure \ref{fig:correlation}, based on the mean precipitation), but both the CP and the VP were included to increase the sample size. There is only a small increasing trend with the complexity of the method, but most of the differences are between reanalyses, with a growing impact for more complex methods. When using moisture variables, ERA-INT, MERRA-2, and CERA-20C were slightly superior to the others.

The impact of the reanalyses on the biases were assessed on the first analogue and not on the mean of the analogue values, to avoid inducing a dry bias for high precipitation values. Considering only the first analogue value is not recommended for the use of AMs in impact models, and an approach such as Schaake Shuffle \citep{Clark2004a} would be wiser. However, it was found acceptable for the purpose of comparing reanalyses. Figure \ref{fig:biases} shows that 2SLP induced a dry bias for most reanalyses, and methods based on the preselection on temperatures were not centred on zero in average, with an overall wet bias for PT-2Z-4MI and a dry bias for PT-2Z-4W-4MI. The reanalyses certainly had an impact on the biases, but it was changing from a method to another, and there is no clear tendency. Future use of AMs might rely on Fig. \ref{fig:biases} to guide their choice, according to the predictors selected.


\subsection{Spatial patterns}

The 301 precipitation stations are located at different elevations and are subject to various meteorological influences. In order to analyze spatial patterns of the methods/reanalyses relationship, maps of the best method per reanalysis are presented in Fig. \ref{fig:map_best_methods}. The selection of the optimal method was previously shown to vary with the reanalysis. One can now also see that it is not equivalent for all stations, but there are spatial patterns depending on the local climate. The three most complex methods (4Z-2MI, PT-2Z-4MI, and PT-2Z-4W-4MI) were almost always selected. The PT-2Z-4MI and PT-2Z-4W-4MI methods were developed for a context of large flood plains, and 4Z-2MI in the context of the upper Rhone catchment in Switzerland. There is a tendency in these maps for the methods to be selected as optimal in their original context, respectively in relatively flat plains or an Alpine environment. Moreover, the fact that a variable such as vertical velocity is considered at a low resolution (0.5\degree -- 2.5\degree) may still make sense in large plains as an uplift/subsidence index, but may be less relevant in narrow alpine valleys. 

The variability between the maps is likely related to the predictive skill of the variables from the different reanalyses. Vertical velocity seems to be overall not optimal in 20CR-2c, but preferable in JRA-55(C) and ERA-20C. The choice of the reanalysis and the AM should then take into account the context of the area of interest.


\subsection{Selection of the analogue dates}

The use of a certain reanalysis over another had an influence on the selection of the analogue days. The first difference is the number of selected analogues, as shown in Fig. \ref{fig:number_analogues}, which were optimized for every reanalysis, method and station. Usually, a higher number of analogues indicates lower performances. This is most visible for 20CR-2c, and partly for NR-1 and NR-2. On the contrary, lower number of analogues were not always associated with the most performing reanalyses.

The selected dates were compared between reanalyses for all stations and all AMs. Figure \ref{fig:similarities_analogue_dates} shows the percentage of similar analogue dates selected when using the reanalyses in columns that were also found when using the reanalyses in rows for the different AMs. The values were averaged for all stations on the VP (same results on the CP). One should remember that the methods were calibrated for all stations and all reanalyses independently, and thus the spatial windows on which the predictors were compared might differ.

As expected, more complex AMs showed lower percentages of similar analogue days between the reanalyses. Indeed, higher correspondence is expected for circulation variables than moisture variables, which are more model-dependent. Reanalyses that are similar products, such as NR-1 and NR-2 or JRA-55 and JRA-55C, showed the highest similarity. However, differences still exist although these reanalyses perform almost equally well (Fig. \ref{fig:comparison_relative}). Higher similarities were also observed between CERA-20C and ERA-20C for methods based on circulation, but not significantly for more complex methods. CERA-20C is thus substantially different than ERA-20C.

20CR-2c differed the most from other reanalyses for all methods. This difference in the selection of analogue days led to lower performance of the methods (Fig. \ref{fig:comparison_relative}). However, it is less present row-wise, due to the selection of a higher number of analogues for 20CR-2c (Fig. \ref{fig:number_analogues}), which increased the matching probability. Another noticeable difference is for MERRA-2 and the 2SLP method. In this case, this departure led to better performance scores (Fig. \ref{fig:comparison_relative}). The selection based on JRA-55 and JRA-55C had globally the highest correspondence to the other reanalyses.


\section{Further analyzes}
\label{sec:analyzes}

\subsection{On the spatial resolution}
\label{sec:resolution}

The different reanalyses are characterized by various grid resolutions. Obviously, higher model resolutions allow for better resolving convection, although they are still relatively coarse. What is not so clear however, is the influence of the output grid resolution when using the data in the AM. In order to assess its impact on the methods performance, reanalyses with higher resolution were degraded to increasingly lower resolutions. This was simply performed by skipping points, which provided reduced resolutions as factors of the original one. No more advanced techniques, such as a spectral transformation was considered. For each resolution, the parameters of the AMs were calibrated again, independently for every method, reanalysis and station, and were thus optimal for a given configuration. 

The impact of the resolution degradation is presented in Fig. \ref{fig:plot_impact_resolution} for six AMs and a selection of 30 stations (Fig. \ref{fig:stations}). Degrading the resolution from below to above 1\degree\ had limited impact on the skill of the methods, eventually except for MERRA-2's SLP. The availability of high resolution outputs was not the main factor influencing the skill of AMs, and future increases in output resolutions should not bring substantial improvements. Higher model resolutions might however allow for better representing orographic effects and convection. 

Beyond 1\degree, the decrease in performance was systematic, but not of the same magnitude for every reanalysis and method. As expected, methods relying on Z were less sensitive to the resolution than the ones with moisture variables. With no surprise, the most complex method, PT-2Z-4VV-4MI was in general the most sensitive to the resolution as it relies on more local information. The 4Z method was the less sensitive, also when compared to 2Z on the first iterations. This might be due to a higher number of informative variables on the atmospheric circulation as it relied on more fields of the geopotential height. For this method, even a reduction of the resolution to 2\degree\ had limited impact.


\subsection{On the archives length}
\label{sec:length}

All previous comparisons were performed for the period 1981--2010. However, some reanalyses have the important added value of covering longer periods. Longer reanalyses have mainly two benefits: they allow analysis of past periods, for example to reconstruct the meteorological conditions related to a flood event, and they enrich the pool of potential analogue situations primarily for less frequent situations. The second aspect was the focus of the present analysis. 

\citet{Ruosteenoja1988} and \citet{Vandendool1994} have shown that a longer archive improves the quality of the meteorological analogy. The different AMs were recalibrated on the same CP as previously and assessed on the same VP (Sect. \ref{sec:data}\ref{sec:precip}), but with an increasing archive, which constituted the pool of potential analogue situations, by adding dates (by blocks of 10 to 20 yrs) farther in the past back to 1871 (for 20CR-2c). The influence of the archive's length on the VP is presented in Fig. \ref{fig:plot_impact_length} for five AMs and the NR-1, JRA-55, CERA-20C, and 20CR-2c reanalyses, on the 30 stations with longer precipitation series available (Fig. \ref{fig:stations}). 

As expected, there was an overall improvement with longer archives than the 24 years from the CP. The gain of longer archives for AMs based on the atmospheric circulation only (2Z and 4Z, panel a) was generally superior to other methods with multiple levels of analogy. Figure \ref{fig:plot_impact_length} also shows that the improvement was not constantly growing with the archive's size, and a decrease of the performance could even appear for some reanalyses and methods. NR-1 showed a discontinuity in performance when adding moisture variables from the period 1961--1971, CERA-20C showed a decrease for different methods from 1921 or 1941 backward, and 20CR-2c from 1881 backward.

With perfect predictor and predictand (precipitation) archives, the prediction skill of the different methods would only increase thanks to the enrichment of the pool of potential analogues, up to a certain point where it might flatten out. A decrease in performance can be explained by the presence of less good analogues that degrade the prediction. The presence of less good analogues can be due to (a) the non-preservation of the relationship between predictors and predictands over time, (b) errors in the precipitation archive, or (c) inhomogeneities or errors in the early years of the reanalyses. While it is obvious that the quality of precipitation measurement is not constant over time, and that the climate system presents trends on that period, if these were the main reasons, a break in performance would have appeared at the same time for all reanalyses and eventually methods. The presence of breaks at different years that were reanalysis and variable dependent would suggest that the variability in the predictors quality is likely the leading factor.

NR-1 is known to have significant differences between climatologies prior and after the introduction of satellite data \citep{Kistler2001}, which might explain these discontinuities. CERA-20C and 20CR-2c are more homogeneous in terms of the type of observations that are assimilated, but the number of observations fluctuates over time, resulting in higher variability for the early years. Thus, for periods where measurements were scarce, the models were less constrained to observations and predictors such as moisture, temperature and vertical velocity are more uncertain. The first-guess errors in 20CR-2c are substantially higher prior to 1880 in the Northern Hemisphere due to a lower number of observations \citep{Compo2011}, which corresponds to the break in performance in Fig. \ref{fig:plot_impact_length}. First guess errors or ensemble spreads from a given reanalysis might be used to motivate the choice of an acceptable archive period.


\subsection{On the use of members}
\label{sec:ensemble}

As discussed in the previous section, the reanalyses spanning over the 20$^{th}$ century are more uncertain for the early period. In order to take into account this uncertainty, CERA-20C and 20CR-2c provide 10 and 56 members respectively. These ensemble reanalyses can be used in the analogue method by looking for similar days on every member. Both the target and the candidate situations are thus extracted from the same member. Two options are possible to merge the selected analogues : (a) by keeping all analogue dates including the duplicates, or (b) by keeping only once an analogue date and removing duplicates. For both options the optimal number of analogues needs to be reassessed. If the data from the different members were perfectly identical, the optimized number of analogues of the first approach would be $m$ times higher as the selection on a single member, $m$ being the number of members considered. On the contrary, the number of analogues would not change for the second approach. Both approaches were assessed here for the methods 2Z (Fig. \ref{fig:plot_impact_members_2Z}) and 2Z-2MI (Fig. \ref{fig:plot_impact_members_2Z-2MI}), because their predictors were available for 20CR-2c. As the spread is lower for a recent period than in the past \citep{Compo2011}, two periods were assessed: 1901--1930 and 1981--2010. There might be other benefits in using members, such as a better consideration of the uncertainty when working on the distant past. However, their impact was here only assessed in terms of performance.

The introduction of members slightly improved the performance of the 2Z method, but only/mainly when keeping duplicate dates (Fig. \ref{fig:plot_impact_members_2Z} a and b). Indeed, the exclusion of duplicate dates led to small or even no improvement. The likely reason is that recurring analogues are probably the best ones, and allowing duplicates gives them more weight. Otherwise, their importance would decrease within a growing selection of analogues. Unsurprisingly, the benefit of using members was also higher for the early period 1901--1930 (Fig. \ref{fig:plot_impact_members_2Z} right), where larger uncertainties are present. In most situations, the additional gain in performance brought by new members flattened out relatively rapidly. Indeed, when using 20CR-2c, the increase in skill after 5 members was rather marginal, which was also the case with CERA-20C on the more recent 1981--2010 period. Using all members of 20CR-2c was very costly in terms of processing time and provided no improvement to the performance. 

The results with the 2Z-2MI method (Fig. \ref{fig:plot_impact_members_2Z-2MI}) led to the same conclusions in terms of higher gains when allowing duplicates and for the earlier 1901--1930 period. However, a major difference is that after having reached the optimal number of members (4--5), the performance did not flatten out, but it might have even decreased to lower values than with a single member. This behaviour was investigated and a peculiar evolution of the number of analogues was found. The number of analogues was optimized for each level of analogy when adding new members, by assessing multiple combinations, so that they were optimal for the provided predictors. Here, the optimal number of analogues tended to be equal for both levels after addition of some members, which means that the selection of the second level of analogy on moisture was discarded. This behaviour did not happen when adding real data from the past (Sect. \ref{sec:analyzes}\ref{sec:length}). The uncertainty between the members were not of the same amplitude for the different variables. A likely hypothesis is that moisture variables being more uncertain, their related number of analogues were growing faster than for Z, but were limited by the selection on the first level of analogy. A great caution is then advised when using AMs with multiple analogy levels on ensemble reanalyses.


\section{Discussion and conclusion}

Some constraints might drive the choice of a certain reanalysis over another, for example when working on earlier periods. However, when the period of interest falls within the satellite era, one has to choose a dataset amongst all these supposedly equivalent reanalyses. The choice is often motivated by either ease of access (availability of the dataset at the institution), ease of use (availability of code to read it), or  preferences for the local provider (such as ECMWF for Europe). This choice has a non negligible impact, which was quantified in this work.

Although compared on a recent period over a data-rich region, the tested reanalyses resulted in large differences in terms of the performance of the AMs. The impact of the reanalyses was found to be sometimes even larger than the choice of the method and its related predictors, in accordance with \citet{Dayon2015}. There was no single overall winner, but different alternatives that provided similar performances. The impact on the skill of AMs is not a direct assessment of the quality of the reanalysis, but it characterizes an indirect impact on the quality of the relationship between predictors and the precipitation, which makes it complex to interpret. However, provided the results obtained, it seems manifest that there is indeed a link between the quality of a reanalysis and its impact on the skill of the AMs.

Figure \ref{fig:synthesis-table} synthesizes the suggested choice of reanalyses for different periods and variables, providing the preferable reanalyses and their alternatives. It is by no means encouraged to change of reanalysis when working on a long period, as homogeneity is of paramount importance. The periods in Fig. \ref{fig:synthesis-table} are to be interpreted as starting periods for the planned study, and time coverage should be checked in Table \ref{table:datasets}. The temporal homogeneity of the reanalyses was not fully assessed here, and users should consider this aspect depending on the application. The different reanalyses are discussed hereafter.

NR-1 and NR-2 were the first reanalyses available and were used for a long time, until recently. Despite of their age and all the progress made in terms of data assimilation and numerical modelling since their production, they still provided valuable outputs. However, they were systematically performing slightly below average, and are thus not of particular interest compared to other options. Even though they start in 1948, which is prior to other reanalyses, there are better alternatives, and we do not recommend using them exclusively any more.

ERA-INT is nowadays often the default choice in Europe for various applications. It was found here to be amongst the best performing reanalyses, except for 2SLP and particularly 2Z, where the skill was substantially lower. The reason for lower skill of the circulation-only predictors was not investigated and might be addressed in the coming ERA-5. However, for circulation-only AMs, it might be safer to consider another reanalysis. The use of moisture variables from ERA-INT in AMs led to particularly higher inter-annual correlation, in competition with MERRA-2, which makes it a dataset of choice for moisture variables.

The new NCEP reanalysis, CFSR, systematically surpassed its predecessors NR-1 and NR-2. It was in the top selection except for W, where it did not perform as well as other options.

The two Japanese reanalyses, JRA-55 and JRA-55C, are less known, but they resulted in remarkably good overall performances and are systematically in the first choice or alternative selection (Fig. \ref{fig:synthesis-table}). A striking element is the similar performance of both reanalyses, despite the fact that JRA-55C only assimilates conventional observations. This suggests that the forecast model used has good physics. Indeed, the forecast model is expected to have a considerable impact on the quality of a reanalysis \citep{Kobayashi2015}, particularly when assimilated observations is sparse. JRA-55C is the recommended reanalysis when the working period starts prior to the satellite era (from 1958 onward), as it is expected to be more homogeneous than JRA-55 due to its use of conventional-only data, and as it performed equally well.

20CR-2c is the only reanalysis so far providing data for the second half of the 19$^{th}$ century, which makes it a valuable asset. However, when working on more recent periods, its performance on the daily precipitation results was systematically and substantially inferior to the other reanalyses. Although it sometimes showed inter-annual correlations at the same level as the other reanalyses, its overall lower performance at a daily time step discards it as an option for other periods than the far past. Its lower performance in the AM was also raised by \citet[][]{Dayon2015}, particularly when local predictors are included. It can be at least partly explained by the fact that 20CR-2c assimilates surface pressure only, and the other variables are thus unadjusted model outputs. Thus, they can be expected to be more influenced by the model than the observations. Nevertheless, it is noteworthy to mention all the informative outputs generated over such a long period on the basis of so little assimilated data.

ERA-20C assimilates marine wind observations in addition to the data ingested in 20CR-2c, and the model is also forced by more data for its boundary conditions. This, along with a different model and assimilation technique, resulted in higher skills than 20CR-2c within the AM. However, ERA-20C did not compete at a daily time step on more recent periods with the other reanalyses that assimilate more observations. CERA-20C has an additional coupling to the ocean and is processed with a more recent version of the IFS forecast model. This resulted in relatively equivalent skills at a daily time step, but higher inter-annual correlations. CERA-20C should then be chosen over ERA-20C.

Finally, MERRA-2 showed a striking performance with SLP, that was as skilful as using 4 geopotential heights. It is difficult to pinpoint the reason for such a better performance, as the assimilated data are essentially the same as the other reanalyses (R. Gelaro, pers. comm., November 16, 2017). A possible reason is the use of a smoothed topography (M. G. Bosilovich, pers. comm., November 14, 2017) or eventually the use of a finite-volume dynamical core (R. Gelaro, pers. comm., November 16, 2017). MERRA-2's SLP seemed to be also more informative at its full high spatial resolution (Sect. \ref{sec:analyzes}\ref{sec:resolution}). When using other variables, it was systematically in the top selection, also in terms of inter-annual correlation.

The differences in skill from a reanalysis to another did not depend on the assimilation technique (at least between 3D-Var and 4D-Var), but on the assimilated data and on the forecast model. Although higher spatial resolutions in the forecast models are likely to result in better reanalyses thanks to a better resolution of convection and orographic effects, higher output resolutions were not found to be determining to explain the differences in skill between reanalyses (Sect. \ref{sec:analyzes}\ref{sec:resolution}). 

Longer archives are commonly considered to improve the analogy by providing more candidate analogues. However, as shown in Sect. \ref{sec:analyzes}\ref{sec:length}, it was not always the case when adding years from a more distant past. One should consider the temporal homogeneity of the archive and the reliability of the variables considered in earlier years. First guess errors or ensemble spreads from a given reanalysis might be used to motivate the choice of an acceptable archive period. As expected, geopotential heights showed a greater robustness over time than moisture variables. 

Some reanalyses provide multiple members, which is an added value for many applications. However, no substantial improvement of the skill was found when using ensemble reanalyses in the AM, at least for recent periods. Moreover, using multiple members in AMs with multiple levels of analogy might even reduce the performance of the method, possibly due to mismatches between the variability of the considered variables. We recommend not using ensembles in the AM for present periods and to use them with great caution for past periods. When using AMs in operational forecasting, the use of forecast ensembles to characterize the target date is however valuable, due to the great uncertainties related to the unknown evolution of the meteorological situation.

Hopefully, the present work can help driving a decision for future use of reanalyses in AMs. The assessment focused on Switzerland only, but it can be expected for the results to be transposable to other data-rich regions, at least in Europe. Indeed, Switzerland has a rich climate with multiple meteorological influences, and the impact of the reanalyses did not vary substantially from one climatic region to another. Moreover, the quality of a reanalysis over space is closely related to the number of assimilated observations, which are relatively dense over Western Europe. For use of AMs in a different context, for example in a data-poor region of the Southern Hemisphere, similar comparative work can be achieved. The present work can still, however, help reducing the number of reanalyses considered.

Finally, instead of choosing a single reanalysis, it can be wise to use several reanalyses, eventually as an ensemble. The most recent products of different institutions should be considered for this kind of approach.





%%%%%%%%%%%%%%%%%%%%%%%%%%%%%%%%%%%%%%%%%%%%%%%%%%%%%%%%%%%%%%%%%%%%%
%  ACKNOWLEDGMENTS
%%%%%%%%%%%%%%%%%%%%%%%%%%%%%%%%%%%%%%%%%%%%%%%%%%%%%%%%%%%%%%%%%%%%%
%
\acknowledgments
The authors want to thank M. Rohrer, P. Laloyaux, R. Gelaro, and M. G. Bosilovich for their valuable inputs.

Precipitation time series were provided by MeteoSwiss. The NCEP/NCAR, NCEP/DOE, and 20CR-2c were provided by the NOAA/OAR/ESRL PSD, Boulder, Colorado, USA, at http://www.esrl.noaa.gov/psd/. Support for the Twentieth Century Reanalysis Project dataset is provided by the U.S. Department of Energy, Office of Science Innovative and Novel Computational Impact on Theory and Experiment (DOE INCITE) program, and Office of Biological and Environmental Research (BER), and by the National Oceanic and Atmospheric Administration Climate Program Office. The CFSR, and JRA-55 were obtained from the CISL Research Data Archive (http://rda.ucar.edu/) at NCAR in Boulder, Colorado, and the NCAR is supported by grants from the National Science Foundation. The Climate Forecast System Reanalysis (CFSR) project is carried out by the Environmental Modeling Center (EMC), National Centers for Environmental Prediction (NCEP). The Japanese 55-year Reanalysis (JRA-55) project is carried out by the Japan Meteorological Agency (JMA). The MERRA-2 was obtained from the Goddard Earth Sciences Data and Information Services Center, Greenbelt, Maryland, from their website at http://disc.sci.gsfc.nasa.gov/mdisc. The ERA-interim, ERA-20C, and CERA-20C were obtained from the ECMWF Data Server at http://apps.ecmwf.int/datasets/. 

Calculations were performed on UBELIX (http://www.id.unibe.ch/hpc), the HPC cluster at the University of Bern. All calculations were performed with the open source AtmoSwing software v1.5.0 \citep{Horton2017a}.


%%%%%%%%%%%%%%%%%%%%%%%%%%%%%%%%%%%%%%%%%%%%%%%%%%%%%%%%%%%%%%%%%%%%%
%  APPENDIXES
%%%%%%%%%%%%%%%%%%%%%%%%%%%%%%%%%%%%%%%%%%%%%%%%%%%%%%%%%%%%%%%%%%%%%
%
% Use \appendix if there is only one appendix.
%\appendix

% Use \appendix[A], \appendix}[B], if you have multiple appendixes.
%\appendix[A]

%% Appendix title is necessary! For appendix title:
%\appendixtitle{}

%%% Appendix section numbering (note, skip \section and begin with \subsection)
% \subsection{First primary heading}

% \subsubsection{First secondary heading}

% \paragraph{First tertiary heading}

%% Important!
%\appendcaption{<appendix letter and number>}{<caption>} 
%must be used for figures and tables in appendixes, e.g.,
%
%\begin{figure}
%\noindent\includegraphics[width=19pc,angle=0]{figure01.pdf}\\
%\appendcaption{A1}{Caption here.}
%\end{figure}
%
% All appendix figures/tables should be placed in order AFTER the main figures/tables, i.e., tables, appendix tables, figures, appendix figures.
%
%%%%%%%%%%%%%%%%%%%%%%%%%%%%%%%%%%%%%%%%%%%%%%%%%%%%%%%%%%%%%%%%%%%%%
%  REFERENCES
%%%%%%%%%%%%%%%%%%%%%%%%%%%%%%%%%%%%%%%%%%%%%%%%%%%%%%%%%%%%%%%%%%%%%
% Make your BibTeX bibliography by using these commands:
\bibliographystyle{ametsoc2014}
\bibliography{../../../Biblio/Mendeley/Export/_For_articles-2017_JCli_reanalysis}


%%%%%%%%%%%%%%%%%%%%%%%%%%%%%%%%%%%%%%%%%%%%%%%%%%%%%%%%%%%%%%%%%%%%%
%  TABLES
%%%%%%%%%%%%%%%%%%%%%%%%%%%%%%%%%%%%%%%%%%%%%%%%%%%%%%%%%%%%%%%%%%%%%
%% Enter tables at the end of the document, before figures.
%%
%
%\begin{table}[t]
%\caption{This is a sample table caption and table layout.  Enter as many tables as
%  necessary at the end of your manuscript. Table from Lorenz (1963).}\label{t1}
%\begin{center}
%\begin{tabular}{ccccrrcrc}
%\hline\hline
%$N$ & $X$ & $Y$ & $Z$\\
%\hline
% 0000 & 0000 & 0010 & 0000 \\
% 0005 & 0004 & 0012 & 0000 \\
% 0010 & 0009 & 0020 & 0000 \\
% 0015 & 0016 & 0036 & 0002 \\
% 0020 & 0030 & 0066 & 0007 \\
% 0025 & 0054 & 0115 & 0024 \\
%\hline
%\end{tabular}
%\end{center}
%\end{table}


\begin{table*}[t]
	\caption{Considered analogue methods with increasing complexity. P0 is the preselection (PC: on calendar basis, that is $\pm 60$ days around the target date), L1, L2 and L3 are the subsequent levels of analogy. The meteorological variables are: SLP -- mean sea level pressure, Z -- geopotential height, T -- air temperature, W -- vertical velocity, MI -- moisture index, which is the product of the relative humidity at the given pressure level and the total water column. The analogy criterion is S1 for SLP and Z and RMSE for the other variables.}
	\begin{center}
		\begin{tabular}{cccccl}
			\hline
			\textbf{Method} & \textbf{P0} & \textbf{L1} & \textbf{L2} & \textbf{L3} & \textbf{Reference} \\ 
			\hline 
			\multirow{2}{*}{\textbf{2SLP}} & \multirow{2}{*}{PC} & SLP@12h &&& \\
			&& SLP@24h &&& \\
			\hline 
			\multirow{2}{*}{\textbf{2Z}} & \multirow{2}{*}{PC} & Z1000@12h &&& \multirow{2}{*}{\citealt{Bontron2004}} \\
			 && Z500@24h &&& \\
	 		\hline 
			\multirow{4}{*}{\textbf{4Z}} & \multirow{4}{*}{PC} & Z1000@06h &&& \multirow{4}{*}{\citealt{Horton2017b}} \\
			 && Z1000@30h &&& \\
			 && Z700@24h &&& \\
			 && Z500@12h &&& \\
			\hline 
			\multirow{2}{*}{\textbf{2Z-2MI}} & \multirow{2}{*}{PC} & Z1000@12h & \multirow{2}{*}{MI850@12+24h} && \multirow{2}{*}{\citealt{Bontron2004}} \\
			&& Z500@24h &&& \\
			\hline 
			\multirow{4}{*}{\textbf{4Z-2MI}} & \multirow{4}{*}{PC} & Z1000@30h &&& \multirow{4}{*}{\citealt{Horton2017b}}\\
			&& Z850@12h & MI700@24h && \\
			&& Z700@24h & MI600@12h && \\
			&& Z400@12h &&& \\
			\hline 
			\multirow{2}{*}{\textbf{PT-2Z-4MI}} & T925@36h & Z1000@12h & MI925@12+24h && \multirow{2}{*}{\citealt{BenDaoud2016}} \\
			& T600@12h & Z500@24h & MI700@12+24h && \\
			\hline 
			\multirow{2}{*}{\textbf{PT-2Z-4W-4MI}} & T925@36h & Z1000@12h & \multirow{2}{*}{W850@06-24h} & MI925@12+24h & \multirow{2}{*}{\citealt{BenDaoud2016}} \\
			& T600@12h & Z500@24h && MI700@12+24h & \\
			\hline 
			
		\end{tabular} 
	\end{center}
	\label{table:methods}
\end{table*}



\begin{table*}[t]
	\caption{Assessed reanalysis datasets with their respective properties, sorted by model age.}
	\begin{center}
		\begin{tabular}{ccccccc}
			\hline
			\multirow{2}{*}{\textbf{Name}} & \multirow{2}{*}{\textbf{Institution}} & \textbf{Period} & \textbf{Output} & \textbf{Model} & \textbf{Model} & \textbf{Assimilation}\\ 
			&& \textbf{of record} & \textbf{resolution} & \textbf{resolution} & \textbf{vintage} & \textbf{technique} \\ 
			\hline 
			\textbf{NR-1} & NCEP, NCAR & 1948 -- present & 2.5\degree x 2.5\degree & T62 ($\sim$1.88\degree), L28 & 1995 & 3D-Var\\
			\textbf{NR-2} & NCEP, DOE & 1948 -- present & 2.5\degree x 2.5\degree & T62 ($\sim$1.88\degree), L28 & 2001 & 3D-Var\\
			\textbf{ERA-INT} & ECMWF & 1979 -- present & 0.75\degree x 0.75\degree & TL255 ($\sim$0.70\degree), L60 & 2006 & 4D-Var\\
			\textbf{CFSR} & NCEP & 1979 -- present & 0.5\degree x 0.5\degree & T382 ($\sim$0.31\degree), L64 & 2009 & 3D-Var\\
			\textbf{JRA-55}  & JMA & 1958 -- present & 1.25\degree x 1.25\degree & TL319 ($\sim$0.36\degree), L60 & 2009 & 4D-Var\\
			\textbf{JRA-55C}  & JMA & 1958 -- 2015 & 1.25\degree x 1.25\degree & TL319 ($\sim$0.36\degree), L60 & 2009 & 4D-Var\\
			\textbf{20CR-2c} & NOAA-CIRES & 1851 -- 2014 & 2\degree x 2\degree & T62 ($\sim$1.88\degree), L28 & 2009 & EnKF\\
			\textbf{ERA-20C} & ECMWF & 1900 -- 2010 & 1\degree x 1\degree & TL159 ($\sim$1.13\degree), L91 & 2012 & 4D-Var\\
			\textbf{MERRA-2} & NASA GMAO & 1980 -- present & 0.625\degree x 0.5\degree & 0.625\degree x 0.5\degree, L72 & 2014 & 3D-Var\\
			\textbf{CERA-20C} & ECMWF & 1901 -- 2010 & 1\degree x 1\degree & T159 ($\sim$1.13\degree), L91 & 2016 & 4D-Var\\
			\hline 
		\end{tabular} 
	\end{center}
	\label{table:datasets}
\end{table*}







%%%%%%%%%%%%%%%%%%%%%%%%%%%%%%%%%%%%%%%%%%%%%%%%%%%%%%%%%%%%%%%%%%%%%
%  FIGURES
%%%%%%%%%%%%%%%%%%%%%%%%%%%%%%%%%%%%%%%%%%%%%%%%%%%%%%%%%%%%%%%%%%%%%
%% Enter figures at the end of the document, after tables.
%%
%
%\begin{figure}[t]
%  \noindent\includegraphics[width=19pc,angle=0]{figure01.pdf}\\
%  \caption{Enter the caption for your figure here.  Repeat as
%  necessary for each of your figures. Figure from \protect\cite{Knutti2008}.}\label{f1}
%\end{figure}

\begin{figure}[t]
  \noindent\includegraphics[width=19pc,angle=0]{figures/map-stations.jpg}\\
  \caption{Map of the 301 precipitation stations with a good data coverage of the period 1981--2010 (blue dots), and the 30 stations with long archives (orange). Background map: \textcopyright\ SwissTopo.}
	\label{fig:stations}
\end{figure}

\begin{figure*}[t]
	\noindent\includegraphics[width=39pc,angle=0]{figures/boxplot-per-method.pdf}\\
	\caption{CRPSS score for all stations and for all considered AMs and reanalysis datasets on the VP. A higher CRPSS score means better performance. The parameters of the AMs were calibrated for every station, every dataset, and every method. The boxes show the 25th, 50th, and 75th percentiles. The whiskers extend to the most extreme data point which is no more than 1.5 times the interquartile range.}
	\label{fig:comparison_values}
\end{figure*}

\begin{figure*}[t]
	\noindent\includegraphics[width=39pc,angle=0]{figures/boxplot-per-method-relative.pdf}\\
	\caption{Impact of the reanalysis dataset on the performance isolated by processing the improvement in CRPSS for one dataset compared to the mean performance on all datasets, per station and per method. Note that the methods cannot be compared here, only the datasets. Same conventions as Fig. \ref{fig:comparison_values}.}
	\label{fig:comparison_relative}
\end{figure*}

\begin{figure*}[t]
	\noindent\includegraphics[width=39pc,angle=0]{figures/boxplot-correl-interannual-mean.pdf}\\
	\caption{Inter-annual correlation between the mean precipitation from the selected analogues and the observations for all stations and for all considered AMs and reanalysis datasets on both the CP and the VP. Same conventions as Fig. \ref{fig:comparison_values}.}
	\label{fig:correlation}
\end{figure*}

\begin{figure*}[t]
	\noindent\includegraphics[width=39pc,angle=0]{figures/boxplot-biases-1st.pdf}\\
	\caption{Same as Fig. \ref{fig:correlation} but for the relative biases.}
	\label{fig:biases}
\end{figure*}

\begin{figure*}[t]
	\noindent\includegraphics[width=38pc,angle=0]{figures/maps-best-methods.pdf}\\
	\caption{Best method per station for the different datasets. NR-2 and JRA-55C are not shown as they are similar to NR-1 and JRA-55 respectively. Background map: \textcopyright\ SwissTopo.}
	\label{fig:map_best_methods}
\end{figure*}

\begin{figure}[t]
	\noindent\includegraphics[width=19pc,angle=0]{figures/number-analogues.pdf}\\
	\caption{Density plots of the optimal number of analogues of the last analogy level for the different AMs and datasets.}
	\label{fig:number_analogues}
\end{figure}

\begin{figure}[t]
	\noindent\includegraphics[width=19pc,angle=0]{figures/similarities-analogue-dates.pdf}\\
	\caption{Percentage of similar analogue dates selected when using the reanalysis datasets in columns that are also found when using the datasets in rows for different AMs. The values are averaged for all stations on the VP.}
	\label{fig:similarities_analogue_dates}
\end{figure}

\begin{figure}[t]
	\noindent\includegraphics[width=19pc,angle=0]{figures/plot-impact-resolution-spread.pdf}\\
	\caption{Impact (difference in CRPSS) of a decrease in grid resolution (degrees) for different datasets and AMs on the CP. The line represents the median and the shaded area represents the first and the third quartiles (on 30 stations).}
	\label{fig:plot_impact_resolution}
\end{figure}

\begin{figure}[t]
	\noindent\includegraphics[width=19pc,angle=0]{figures/plot-impact-length-spread-VP.pdf}\\
	\caption{Impact (difference in CRPSS) on the VP of an increase in the archive length (years) for different datasets and AMs. Results for the 4Z method are displayed along with the 2Z method, but with dashed lines. The line represents the median and the shaded area represents the first and the third quartiles (on 30 stations).}
	\label{fig:plot_impact_length}
\end{figure}

\begin{figure}[t]
	\noindent\includegraphics[width=19pc,angle=0]{figures/plot-impact-members-2Z.pdf}\\
	\caption{Impact (difference in CRPSS) of an increase in the number of ensemble members used for the 2Z method and for CERA-20C and 20CR-2c datasets. The results are provided for two periods: (a, c) 1981--2010 and (b, d) 1901--1930. Two approaches were assessed: (a, b) the first allowing duplicate analogue dates ("w.d.d.") and (c, d) the second without duplicate analogue dates ("wo.d.d."). The line represents the median and the shaded area represents the first and the third quartiles (on 30 stations). The dashed line and striped area correspond to results on the VP. All 56 members of 20CR-2c were assessed and the tendencies continue, but the plots are split at 30 members.}
	\label{fig:plot_impact_members_2Z}
\end{figure}

\begin{figure}[t]
	\noindent\includegraphics[width=19pc,angle=0]{figures/plot-impact-members-2Z-2MI.pdf}\\
	\caption{Same as Fig. \ref{fig:plot_impact_members_2Z} but for the 2Z-2MI method.}
	\label{fig:plot_impact_members_2Z-2MI}
\end{figure}

\begin{figure}[t]
	\noindent\includegraphics[width=19pc,angle=0]{figures/synthesis-table.pdf}\\
	\caption{Synthesis table of the recommended reanalyses to use in AMs for different periods and variables. This recommendation applies to Europe and eventually other data-rich regions of the world. The darker shaded area represent the first choice and the lighter shaded area represents alternatives. When a reanalysis is not mentioned, it is either not available or not recommended.}
	\label{fig:synthesis-table}
\end{figure}

\end{document}