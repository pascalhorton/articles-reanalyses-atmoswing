%% Version 4.3.2, 25 August 2014
%
%%%%%%%%%%%%%%%%%%%%%%%%%%%%%%%%%%%%%%%%%%%%%%%%%%%%%%%%%%%%%%%%%%%%%%
% Template.tex --  LaTeX-based template for submissions to the 
% American Meteorological Society
%
% Template developed by Amy Hendrickson, 2013, TeXnology Inc., 
% amyh@texnology.com, http://www.texnology.com
% following earlier work by Brian Papa, American Meteorological Society
%
% Email questions to latex@ametsoc.org.
%
%%%%%%%%%%%%%%%%%%%%%%%%%%%%%%%%%%%%%%%%%%%%%%%%%%%%%%%%%%%%%%%%%%%%%
%  PREAMBLE
%%%%%%%%%%%%%%%%%%%%%%%%%%%%%%%%%%%%%%%%%%%%%%%%%%%%%%%%%%%%%%%%%%%%%

%% Start with one of the following:
%  DOUBLE-SPACED VERSION FOR SUBMISSION TO THE AMS
\documentclass{ametsoc}

%  TWO-COLUMN JOURNAL PAGE LAYOUT---FOR AUTHOR USE ONLY
%\documentclass[twocol]{ametsoc}

\usepackage{multirow}
\usepackage{gensymb}

%%%%%%%%%%%%%%%%%%%%%%%%%%%%%%%%
%%% To be entered only if twocol option is used

\journal{jcli}

%  Please choose a journal abbreviation to use above from the following list:
% 
%   jamc     (Journal of Applied Meteorology and Climatology)
%   jtech     (Journal of Atmospheric and Oceanic Technology)
%   jhm      (Journal of Hydrometeorology)
%   jpo     (Journal of Physical Oceanography)
%   jas      (Journal of Atmospheric Sciences)	
%   jcli      (Journal of Climate)
%   mwr      (Monthly Weather Review)
%   wcas      (Weather, Climate, and Society)
%   waf       (Weather and Forecasting)
%   bams (Bulletin of the American Meteorological Society)
%   ei    (Earth Interactions)

%%%%%%%%%%%%%%%%%%%%%%%%%%%%%%%%
%Citations should be of the form ``author year''  not ``author, year''
\bibpunct{(}{)}{;}{a}{}{,}

%%%%%%%%%%%%%%%%%%%%%%%%%%%%%%%%

%%% To be entered by author:

%% May use \\ to break lines in title:

\title{Comparison of present reanalysis datasets in the context of the analogue method for statistical precipitation downscaling.}

%%% Enter authors' names, as you see in this example:
%%% Use \correspondingauthor{} and \thanks{Current Affiliation:...}
%%% immediately following the appropriate author.
%%%
%%% Note that the \correspondingauthor{} command is NECESSARY.
%%% The \thanks{} commands are OPTIONAL.

    %\authors{Author One\correspondingauthor{Author One, 
    % American Meteorological Society, 
    % 45 Beacon St., Boston, MA 02108.}
% and Author Two\thanks{Current affiliation: American Meteorological Society, 
    % 45 Beacon St., Boston, MA 02108.}}

\authors{Pascal Horton\correspondingauthor{University of Bern, Institute of Geography, Hallerstrasse 12, 3012 Bern, Switzerland.}, Rolf Weingartner, Stefan Br\"{o}nnimann}

%% Follow this form:
    % \affiliation{American Meteorological Society, 
    % Boston, Massachusetts.}

\affiliation{Oeschger Centre for Climate Change Research, Institute of Geography, University of Bern, Bern, Switzerland}

%% Follow this form:
    %\email{latex@ametsoc.org}

\email{pascal.horton@giub.unibe.ch}

%% If appropriate, add additional authors, different affiliations:
    %\extraauthor{Extra Author}
    %\extraaffil{Affiliation, City, State/Province, Country}

\extraauthor{Charles Obled}
\extraaffil{Laboratoire d'\'{e}tude des Transferts en Hydrologie et Environnement (LTHE), Universit\'{e} de Grenoble-Alpes, Grenoble, France}


%%%%%%%%%%%%%%%%%%%%%%%%%%%%%%%%%%%%%%%%%%%%%%%%%%%%%%%%%%%%%%%%%%%%%
%  ABSTRACT
%
% Enter your abstract here
% Abstracts should not exceed 250 words in length!
%
% For BAMS authors only: If your article requires a Capsule Summary, please place the capsule text at the end of your abstract
% and identify it as the capsule. Example: This is the end of the abstract. (Capsule Summary) This is the capsule summary. 

\abstract{The analogue method is a statistical downscaling method for precipitation prediction. It uses similarity in terms of synoptic-scale predictors with situations in the past in order to provide a probabilistic prediction for the day of interest. It has been used for decades in a context of weather or flood forecasting, and is more recently also applied to climate studies, whether for reconstruction of past weather conditions or future climate impact studies. In order to evaluate the relationship between synoptic scale predictors and the local weather variable of interest, e.g. precipitation, reanalysis datasets are necessary. Nowadays, the number of available reanalysis datasets increase. These are generated by different atmospheric models with different assimilation techniques and offer various spatial and temporal resolutions. A major difference between these datasets is also the length of the archive they provide. While some datasets start at the beginning of the satellite era and assimilate these data, others aim at homogeneity on a longer period and only assimilate conventional observations.
The context of the application of analogue methods might drive the choice of an appropriate dataset, for example when the archive length is a leading criterion. However, in many studies, a reanalysis dataset is subjectively chosen, according to the user’s preferences or the ease of access. The impact of this choice on the results of the downscaling procedure is rarely considered and no comprehensive comparison has been undertaken so far.
In order to fill this gap and to advise on the choice of appropriate datasets, nine different global reanalysis datasets were compared in seven distinct versions of analogue methods, over 300 precipitation stations in Switzerland. Significant differences in terms of prediction performance were identified. Although the impact of the reanalysis dataset on the skill score varies according to the chosen predictor, be it atmospheric circulation or thermodynamic variables, some hierarchy between the datasets is often preserved.
}

\begin{document}

%% Necessary!
\maketitle


%%%%%%%%%%%%%%%%%%%%%%%%%%%%%%%%%%%%%%%%%%%%%%%%%%%%%%%%%%%%%%%%%%%%%
%  MAIN BODY OF PAPER
%%%%%%%%%%%%%%%%%%%%%%%%%%%%%%%%%%%%%%%%%%%%%%%%%%%%%%%%%%%%%%%%%%%%%
%

%% In all cases, if there is only one entry of this type within
%% the higher level heading, use the star form: 
%%
% \section{Section title}
% \subsection*{subsection}
% text...
% \section{Section title}

%vs

% \section{Section title}
% \subsection{subsection one}
% text...
% \subsection{subsection two}
% \section{Section title}

%%%
% \section{First primary heading}

% \subsection{First secondary heading}

% \subsubsection{First tertiary heading}

% \paragraph{First quaternary heading}


%TODO: publish results as datasets?
%TODO: AM not sensitive to biases

\section{Introduction}

\section{Data and methods}
\label{sec:data}

\subsection{Reanalysis datasets}



NR-1 \citep{Kalnay1996, Kistler2001}
NR-2 \citep{Kanamitsu2002}
ERA-INT \citep{Dee2011a}
CFSR \citep{Saha2010a}
JRA-55 \citep{Kobayashi2015, Harada2016}
JRA-55C
20CR-2c \citep{Compo2011}
ERA-20C \citep{Poli2016}
MERRA-2. MERRA-2 is an update for the first MERRA reanalysis \citep{Rienecker2011}
CERA-20C


1981--2010



%https://www.ncbi.nlm.nih.gov/pmc/articles/PMC4461156/
%http://reanalyses.org/comment/3843
%https://disc.gsfc.nasa.gov/datareleases/merra_2_data_release
%https://climatedataguide.ucar.edu/climate-data/era-20c-ecmwfs-atmospheric-reanalysis-20th-century-and-comparisons-noaas-20cr?qt-climatedatasetmaintabs=5#qt-climatedatasetmaintabs
%https://climatedataguide.ucar.edu/climate-data/era-20c-ecmwfs-atmospheric-reanalysis-20th-century-and-comparisons-noaas-20cr
%https://www.esrl.noaa.gov/psd/data/gridded/data.ncep.reanalysis.derived.html
%https://www.esrl.noaa.gov/psd/data/20thC_Rean/
%https://climatedataguide.ucar.edu/climate-data/atmospheric-reanalysis-overview-comparison-tables



\subsection{Precipitation dataset}
\label{sec:precip}

The predictands -- variables to be predicted -- considered here are daily precipitation totals (06:00 h UTC to 06:00 h UTC the following day) at 301 weather stations of the MeteoSwiss network in Switzerland (Fig. \ref{fig:stations}). All stations with a good data coverage of the period 1981--2010 were considered. Most AM applications use gridded precipitation or catchment-scale aggregated series, but any data manipulation were avoided here in order to obviate any undesired interference with the sensitivity analysis. The precipitation data were not transformed by a square root like in some other studies. Thirty stations -- those with longer time series -- were selected among this selection for additional analyses (Sect. \ref{sec:analyzes}).

The 30-year precipitation dataset was divided into a calibration period (CP) and an independent validation period (VP). In order to reduce the impact of potential inhomogeneities in the time series, the selection of the VP was evenly distributed over the entire series \citep{BenDaoud2010}. A total of 6 years was considered for the VP by selecting 1 out of every 5 years. Days from the VP were never used as candidate situations for the selection of analogues.


\subsection{Considered analogue methods}
\label{sec:ams}

Different variations of the analogue method were considered in the present work (Table \ref{table:methods}). These methods have a varying complexity and are constituted of a single or multiple subsequent levels of analogy with predictor variables of different kind. A relatively simple method that is often considered as reference is based on the analogy of synoptic circulation on two geopotential heights (Z1000 at 12:00 h UTC and Z500 at 24 h UTC) and is named here 2Z.

The 2Z method consists of the following steps: first, to cope with seasonal effects, candidate dates are extracted within a period of four months centred around the target date, for every year of the archive (PC: preselection on calendar basis in Table \ref{table:methods}). Then, the similarity of the atmospheric circulation of a target date with every day of the archive is assessed by processing the S1 criterion \citep[Eq.\ \ref{eq:S1}, ][]{Teweles1954, Drosdowsky2003}, which is a comparison of gradients, over a certain spatial window:

\begin{equation}
\label{eq:S1}
S1=100 \frac {\displaystyle \sum_{i} \vert \Delta\hat{z}_{i} - \Delta z_{i} \vert}
{\displaystyle \sum_{i} max\left\lbrace \vert \Delta\hat{z}_{i} \vert , \vert \Delta z_{i} \vert \right\rbrace }
\end{equation}
where $\Delta \hat{z}_{i}$ is the difference in geopotential height between the \textit{i}-th pair of adjacent points of gridded data describing the target situation, and $\Delta z_{i}$ is the corresponding observed geopotential height difference in the candidate situation. The smaller the S1 values, the more similar the pressure fields. This criteria being processed on gradients, it is not sensitive to biases in the considered dataset, as long as the circulation is correctly represented. \citet{Bontron2004} showed that the S1 criterion performs better than scores based on absolute distances, as it compares the circulation patterns rather than the absolute value of the geopotential height. 

The $N_{1}$ dates with the lowest values of S1 are considered as analogues to the target day. The number of analogues, $N_{1}$, is a parameter to calibrate. Then, the daily observed precipitation amount for the $N_{1}$ selected dates provide the empirical conditional distribution, considered as the probabilistic prediction for the target day. The choice of the predictors and their corresponding pressure levels and temporal windows were optimized for the NR-1 dataset.

A variation of the former method, but based on the mean sea level pressure (2SLP) rather than geopotential heights was also assessed in this work. The S1 criteria was also used to quantify the analogy between the pressure fields.

Another method relying only on the atmospheric circulation has also been considered. It relies on four geopotential heights (4Z, Table \ref{table:methods}) that were automatically selected by genetic algorithms for the upper Rhone catchment in Switzerland \citep{Horton2017b}. The 4Z method was shown to significantly outperform the 2Z method by exploiting more information from the geopotential heights and by taking advantage of additional degrees of freedom, such as different spatial windows between the pressure levels and the introduction of a weighting between them. However, due to the high number of datasets and stations considered in this work, it was not possible to use genetic algorithms in order to optimize the method. Thus, the 4Z method considered here is a simplification of the results from \citet{Horton2017b}, and only the selection of the optimal pressure levels and temporal windows were considered (Z1000 at 06:00 and 30:00 h UTC, Z700 at 24:00 h UTC, and Z500 at 12:00 h UTC), and fixed for all stations. As for the 2Z method, a unique but station-specific spatial window for all pressure level was considered, and the weights between the pressure levels have equal values. Such simplifications of the parameters resulted in a decrease of the performance score, which however was still superior to the one of the 2Z method.

The other methods considered hereafter add a second, or more, subsequent level(s) of analogy after the analogy of the atmospheric circulation. Unlike stated in \citet{Caillouet2016}, stepwise analogue methods existed for some time \citep[e.g.][]{Bontron2004, Bontron2005, Marty2010, Marty2012, Horton2012a}. The next parametrization adds a second level of analogy on the moisture variables (method 2Z-2MI, Table \ref{table:methods}). The predictor that \citet{Bontron2004} found optimal for France is a moisture index (MI) made of the product of the total precipitable water (TPW) with the relative humidity at 850~hPa (RH850). \cite{Horton2012a} confirmed that this index is also better for the Swiss Alps than any other variable from the NR-1 considered independently. When adding a second level of analogy, $N_{2}$ dates are subsampled within the $N_{1}$ analogues of the atmospheric circulation, to end up with a smaller number of analogue situations. When this second level of analogy is added, a higher number of analogues $N_{1}$ is kept on the first level. 

Similarly to the 4Z method, the 4Z-2MI is a simplification of the methods optimized by genetic algorithms in \citet{Horton2017b}. It consists of a first level of analog on four geopotential heights (Z1000 at 30:00 h UTC, Z850 at 12:00 h UTC, Z700 at 24:00 h UTC, and Z400 at 12:00 h UTC) followed by the moisture index (MI) at two pressure levels (MI700 at 24:00 h UTC and MI600 at 12:00 h UTC).

\citet{BenDaoud2016} replaced the calendar preselection ($\pm$ 60 days around the target date) by a preselection on similar air temperature (T925 at 36:00 h UTC and T600 at 12:00 h UTC, at the nearest grid point). It allows a more dynamic screening of similar situations in terms of air masses as the seasonal signal is also present in the temperature data. The undesired mixing of spring and autumn situations is discussed in \citet{Caillouet2016}. The number of preselected dates is equivalent to the number of days one would have chosen with the calendar approach, and thus depends on the archive size: $N_{0} = 120 \cdot n_{a}$ where $n_{a}$ is the number of years in the calibration period. In this method, named PT-2Z-4MI, the analogy of the atmospheric circulation is the same as in the 2Z method, but the moisture analogy differs (MI925 and MI700 at 12:00 and 24:00 h UTC).

Subsequently, \citet{BenDaoud2016} added an additional level of analogy between the circulation and the moisture analogy (PT-2Z-4W-4MI, Table \ref{table:methods}) based on vertical velocity at 850~hPa (W850). This AM was primarily developed for large floodplains in France (Sa\^{o}ne, Seine) and is the most complex method considered in this work. 

AMs rely on parameters that need to be defined for every level of analogy. In this work, the choice of the predictors and their corresponding temporal windows (hour of the day) was identical to the listed methods in Table \ref{table:methods} and were not reassessed. The parameters that were here calibrated for every station, method, and dataset, are listed below.

\begin{itemize}
	\item The spatial windows, which are the domains in which the predictors are compared. A spatial window is specific to each level of analogy and is here shared between the predictors of that level.
	\item The optimal number of analogue situations to sample for every level of analogy.
\end{itemize}

The semi-automatic sequential procedure developed by \citet{Bontron2004} was used to calibrate the AM. The procedure used is described in \citet{Horton2017c} and is similar to the work of \citet{Radanovics2013} and \citet{BenDaoud2016}. It was implemented in the AtmoSwing-optimizer software (www.atmoswing.org), which was used to perform the calibration.


\subsection{Performance assessment}

The CRPS \citep[Continuous Ranked Probability Score,][]{Brown1974, Matheson1976, Hersbach2000} is often used to assess the performance of AMs. It allows evaluating the predicted cumulative distribution functions $F(y)$, for example, of the precipitation values $y$ from analogue situations, compared to the observed value $y^{0}$. The better the prediction, the smaller the score. The mean CRPS of a prediction series of length $n$ can be written as:

\begin{equation}
\label{eq:CRPS}
CRPS = \frac{1}{n} \sum_{i=1}^{n} \left(  \int_{-\infty}^{+\infty} \left[ F_{i}(y)-H_{i}(y-y_{i}^{0})\right]^{2} dy \right) 
\end{equation}
where $H(y-y_{i}^{0})$ is the Heaviside function that is null when $y-y_{i}^{0}<0$, and has the value 1 otherwise.

In order to compare the value of the score relative to a reference, one often considers its skill score expression, and uses the climatological distribution of precipitation from the entire archive as the reference. However, the choice of the reference is not important when comparing performances, since we consider its increase or decrease rather than the CRPSS absolute value. The CRPSS (Continuous Ranked Probability Skill Score) is thus defined as follows:

\begin{equation}
\label{eq:CRPSS}
CRPSS = \frac{CRPS-CRPS_{r}}{CRPS_{p}-CRPS_{r}} = 1-\frac{CRPS}{CRPS_{r}}
\end{equation}
where $CRPS_{r}$ is the CRPS value for the reference and $CRPS_{p}$ would be the one for a perfect prediction (which implies $CRPS_{p}~=~0$). A better prediction is characterized by an increase in CRPSS.

%TODO: other scores !?


\section{Influence of the reanalysis dataset}

All considered AMs were calibrated for every dataset and station, which resulted in a total of 21,070 runs processed on a HPC cluster at the University of Bern. For every combination, the spatial windows and the number of analogues of each analogy level were optimized with a semi-automatic sequential procedure (Sect. \ref{sec:data}\ref{sec:ams}). Thus, the parameters were independent for any configuration, and the number of analogues was always optimal. All results are shown for the VP (Sect. \ref{sec:data}\ref{sec:precip}), and are thus not due to over-parameterization. The scores processed on the CP were similar.

The results of 20CR-2c are shown for the ensemble mean only. The same analyzes were performed on a single member, but no significant difference has been observed. The single-member might be slightly less skillful than the ensemble mean, but only by a small amount (not shown).


\subsection{Mean performance}

The performances (CRPSS score) of all considered AMs and reanalysis datasets are shown in Fig. \ref{fig:comparison_values}. Overall, the skill of the AMs tend to increase with their complexity. The two first methods based on two circulation predictors, 2SLP and 2Z, are equivalent, except for MERRA-2, where SLP show a higher predictive skill than Z. Then, there is a systematic increase of the skill when selecting 2Z, 4Z, 2Z-2MI, or 4Z-2MI. Finally, the respective performance of the 4Z-2MI, PT-2Z-4MI and the PT-2Z-4W-4MI methods vary from one dataset to another. Thus, an increase in complexity of the AM can only be justified for some datasets.

The effect of the reanalysis dataset was isolated in Figure \ref{fig:comparison_relative} by processing the difference in CRPSS for one dataset compared to the mean performance on all datasets, per station and per method. The variability is reduced because the climatological differences between the stations were mostly removed. There is a tendency for the impact of the dataset to increase with the complexity of the method. This is particularly visible for JRA-55, JRA-55C and 20CR-2c. The relative performance of 20CR-2c significantly decreases for more complex methods, even though the absolute skill score generally increases (Fig. \ref{fig:comparison_values}). The 2SLP method was found to be particularly skillful with MERRA-2 compared to other datasets. 

The 301 precipitation stations are located at different heights and are subject to various meteorological influences. In order to analyze spatial patterns of the methods-datasets relationship, maps of the best method per dataset are presented in Fig. \ref{fig:map_best_methods}. The selection of the optimal method was previously shown to vary with the dataset. One can now also see that it is not equivalent for all stations, but there are spatial patterns depending on the local climate. The three most complex methods (4Z-2MI, PT-2Z-4MI, and PT-2Z-4W-4MI) are almost always selected. The PT-2Z-4MI and PT-2Z-4W-4MI methods were developed for a context of large flood plains, and 4Z-2MI in the context of the upper Rhone catchment in Switzerland. There is a tendency in these maps for the methods to be selected as optimal in their original context, respectively the Plateau or the Alps. The variability between the maps is likely related to the predictive skill of the variables from the different datasets. Vertical velocity seems to be overall not optimal in 20CR-2c, but preferable in JRA-55(C) and ERA-20C, which is consistent with Fig. \ref{fig:comparison_values}. The choice of the dataset and the AM should then take into account the context of the area of interest.


\subsection{Performance for different precipitation thresholds}


Fig. \ref{fig:comparison_relative_P0}

Fig. \ref{fig:comparison_relative_Pq99}


\subsection{Selection of the analogue dates}


\section{Further analyzes}
\label{sec:analyzes}

\subsection{Influence of the spatial resolution}

\subsection{Archive length vs ensembles}

\subsection{On the spread of the ensemble in early years}

%TODO: 20CR spread decrease with time -> low for present & high in the past


\subsection{Precipitation data from the reanalysis}


\section{Discussion}


\section{Conclusion}


%%%%%%%%%%%%%%%%%%%%%%%%%%%%%%%%%%%%%%%%%%%%%%%%%%%%%%%%%%%%%%%%%%%%%
%  ACKNOWLEDGMENTS
%%%%%%%%%%%%%%%%%%%%%%%%%%%%%%%%%%%%%%%%%%%%%%%%%%%%%%%%%%%%%%%%%%%%%
%
\acknowledgments
Precipitation time series were provided by MeteoSwiss. The NCEP/NCAR, NCEP/DOE, and 20CR-2c were provided by the NOAA/OAR/ESRL PSD, Boulder, Colorado, USA, at http://www.esrl.noaa.gov/psd/. Support for the Twentieth Century Reanalysis Project dataset is provided by the U.S. Department of Energy, Office of Science Innovative and Novel Computational Impact on Theory and Experiment (DOE INCITE) program, and Office of Biological and Environmental Research (BER), and by the National Oceanic and Atmospheric Administration Climate Program Office. The CFSR, and JRA-55 were obtained from the CISL Research Data Archive (http://rda.ucar.edu/) at NCAR in Boulder, Colorado, and the NCAR is supported by grants from the National Science Foundation. The Climate Forecast System Reanalysis (CFSR) project is carried out by the Environmental Modeling Center (EMC), National Centers for Environmental Prediction (NCEP). The Japanese 55-year Reanalysis (JRA-55) project is carried out by the Japan Meteorological Agency (JMA). The MERRA-2 was obtained from the Goddard Earth Sciences Data and Information Services Center, Greenbelt, Maryland, from their website at http://disc.sci.gsfc.nasa.gov/mdisc. The ERA-interim, ERA-20C, and CERA-20C were obtained from the ECMWF Data Server at http://apps.ecmwf.int/datasets/. 

Calculations were performed on UBELIX (http://www.id.unibe.ch/hpc), the HPC cluster at the University of Bern.


%%%%%%%%%%%%%%%%%%%%%%%%%%%%%%%%%%%%%%%%%%%%%%%%%%%%%%%%%%%%%%%%%%%%%
%  APPENDIXES
%%%%%%%%%%%%%%%%%%%%%%%%%%%%%%%%%%%%%%%%%%%%%%%%%%%%%%%%%%%%%%%%%%%%%
%
% Use \appendix if there is only one appendix.
%\appendix

% Use \appendix[A], \appendix}[B], if you have multiple appendixes.
%\appendix[A]

%% Appendix title is necessary! For appendix title:
%\appendixtitle{}

%%% Appendix section numbering (note, skip \section and begin with \subsection)
% \subsection{First primary heading}

% \subsubsection{First secondary heading}

% \paragraph{First tertiary heading}

%% Important!
%\appendcaption{<appendix letter and number>}{<caption>} 
%must be used for figures and tables in appendixes, e.g.,
%
%\begin{figure}
%\noindent\includegraphics[width=19pc,angle=0]{figure01.pdf}\\
%\appendcaption{A1}{Caption here.}
%\end{figure}
%
% All appendix figures/tables should be placed in order AFTER the main figures/tables, i.e., tables, appendix tables, figures, appendix figures.
%
%%%%%%%%%%%%%%%%%%%%%%%%%%%%%%%%%%%%%%%%%%%%%%%%%%%%%%%%%%%%%%%%%%%%%
%  REFERENCES
%%%%%%%%%%%%%%%%%%%%%%%%%%%%%%%%%%%%%%%%%%%%%%%%%%%%%%%%%%%%%%%%%%%%%
% Make your BibTeX bibliography by using these commands:
\bibliographystyle{ametsoc2014}
\bibliography{../../../Biblio/Mendeley/Export/4_articles-2017_JCli_reanalysis}


%%%%%%%%%%%%%%%%%%%%%%%%%%%%%%%%%%%%%%%%%%%%%%%%%%%%%%%%%%%%%%%%%%%%%
%  TABLES
%%%%%%%%%%%%%%%%%%%%%%%%%%%%%%%%%%%%%%%%%%%%%%%%%%%%%%%%%%%%%%%%%%%%%
%% Enter tables at the end of the document, before figures.
%%
%
%\begin{table}[t]
%\caption{This is a sample table caption and table layout.  Enter as many tables as
%  necessary at the end of your manuscript. Table from Lorenz (1963).}\label{t1}
%\begin{center}
%\begin{tabular}{ccccrrcrc}
%\hline\hline
%$N$ & $X$ & $Y$ & $Z$\\
%\hline
% 0000 & 0000 & 0010 & 0000 \\
% 0005 & 0004 & 0012 & 0000 \\
% 0010 & 0009 & 0020 & 0000 \\
% 0015 & 0016 & 0036 & 0002 \\
% 0020 & 0030 & 0066 & 0007 \\
% 0025 & 0054 & 0115 & 0024 \\
%\hline
%\end{tabular}
%\end{center}
%\end{table}


\begin{table*}[t]
	\caption{Considered analogue methods with increasing complexity. P0 is the preselection (PC: on calendar basis, that is $\pm 60$ days around the target date), L1, L2 and L3 are the subsequent levels of analogy. The meteorological variables are: SLP -- mean sea level pressure, Z -- geopotential height, T -- air temperature, W -- vertical velocity, MI -- moisture index, which is the product of the relative humidity at the given pressure level and the total water column. The analogy criterion is S1 for SLP and Z and RMSE for the other variables.}
	\begin{center}
		\begin{tabular}{cccccl}
			\hline
			\textbf{Method} & \textbf{P0} & \textbf{L1} & \textbf{L2} & \textbf{L3} & \textbf{Reference} \\ 
			\hline 
			\multirow{2}{*}{\textbf{2SLP}} & \multirow{2}{*}{PC} & SLP@12h &&& \\
			&& SLP@24h &&& \\
			\hline 
			\multirow{2}{*}{\textbf{2Z}} & \multirow{2}{*}{PC} & Z1000@12h &&& \multirow{2}{*}{\citealt{Bontron2004}} \\
			 && Z500@24h &&& \\
	 		\hline 
			\multirow{4}{*}{\textbf{4Z}} & \multirow{4}{*}{PC} & Z1000@06h &&& \multirow{4}{*}{\citealt{Horton2017b}} \\
			 && Z1000@30h &&& \\
			 && Z700@24h &&& \\
			 && Z500@12h &&& \\
			\hline 
			\multirow{2}{*}{\textbf{2Z-2MI}} & \multirow{2}{*}{PC} & Z1000@12h & \multirow{2}{*}{MI850@12+24h} && \multirow{2}{*}{\citealt{Bontron2004}} \\
			&& Z500@24h &&& \\
			\hline 
			\multirow{4}{*}{\textbf{4Z-2MI}} & \multirow{4}{*}{PC} & Z1000@30h &&& \multirow{4}{*}{\citealt{Horton2017b}}\\
			&& Z850@12h & MI700@24h && \\
			&& Z700@24h & MI600@12h && \\
			&& Z400@12h &&& \\
			\hline 
			\multirow{2}{*}{\textbf{PT-2Z-4MI}} & T925@36h & Z1000@12h & MI925@12+24h && \multirow{2}{*}{\citealt{BenDaoud2016}} \\
			& T600@12h & Z500@24h & MI700@12+24h && \\
			\hline 
			\multirow{2}{*}{\textbf{PT-2Z-4W-4MI}} & T925@36h & Z1000@12h & \multirow{2}{*}{W850@06-24h} & MI925@12+24h & \multirow{2}{*}{\citealt{BenDaoud2016}} \\
			& T600@12h & Z500@24h && MI700@12+24h & \\
			\hline 
			
		\end{tabular} 
	\end{center}
	\label{table:methods}
\end{table*}



\begin{table*}[t]
	\caption{Assessed reanalysis datasets with their respective available period, timestep and resolution.}
	\begin{center}
		\begin{tabular}{ccccccc}
			\hline
			\multirow{2}{*}{\textbf{Id}} & \multirow{2}{*}{\textbf{Institutions}} & \multirow{2}{*}{\textbf{Name}} & \textbf{Period} & \textbf{Output} & \textbf{Model} & \textbf{Assimilation}\\ 
			&&& \textbf{of record} & \textbf{resolution} & \textbf{vintage} & \textbf{technique} \\ 
			\hline 
			\textbf{NR-1} & NCEP, NCAR & Reanalysis I & 1948 -- present & 2.5\degree x 2.5\degree & 1995 & 3D-Var\\
			\textbf{NR-2} & NCEP, DOE & Reanalysis II & 1948 -- present & 2.5\degree x 2.5\degree & 2001 & 3D-Var\\
			\textbf{ERA-INT} & ECMWF & ERA-Interim & 1979 -- present & 0.75\degree x 0.75\degree & 2006 & 4D-Var\\
			\textbf{CFSR} & NCEP & Clim. Forecast Sys. Rean. & 1979 -- present & 0.5\degree x 0.5\degree & 2009 & 3D-Var\\
			\textbf{JRA-55}  & JMA & Japanese 55-year Rean. & 1958 -- 2015 & 1.25\degree x 1.25\degree & 2009 & 4D-Var\\
			\textbf{JRA-55C}  & JMA & JRA-55 Conventional Data & 1958 -- 2015 & 1.25\degree x 1.25\degree & 2009 & 4D-Var\\
			\textbf{20CR-2c} & NOAA-CIRES & 20th Century Rean. V2c & 1851 -- 2014 & 2\degree x 2\degree & 2009 & E. K. filter\\
			\textbf{ERA-20C} & ECMWF & ERA 20th century & 1900 -- 2011 & 1\degree x 1\degree & 2012 & 4D-Var\\
			\textbf{MERRA-2} & NASA & MERRA-2 & 1980 -- present & 0.625\degree x 0.5\degree & 2014 & 3D-Var\\
			\textbf{CERA-20C} & ECMWF & CERA-20C & 1901 -- 2010 & 1\degree x 1\degree & 2016 & 4D-Var\\
			\hline 
		\end{tabular} 
	\end{center}
	\label{table:datasets}
\end{table*}







%%%%%%%%%%%%%%%%%%%%%%%%%%%%%%%%%%%%%%%%%%%%%%%%%%%%%%%%%%%%%%%%%%%%%
%  FIGURES
%%%%%%%%%%%%%%%%%%%%%%%%%%%%%%%%%%%%%%%%%%%%%%%%%%%%%%%%%%%%%%%%%%%%%
%% Enter figures at the end of the document, after tables.
%%
%
%\begin{figure}[t]
%  \noindent\includegraphics[width=19pc,angle=0]{figure01.pdf}\\
%  \caption{Enter the caption for your figure here.  Repeat as
%  necessary for each of your figures. Figure from \protect\cite{Knutti2008}.}\label{f1}
%\end{figure}

\begin{figure}[t]
  \noindent\includegraphics[width=19pc,angle=0]{figures/map-stations.jpg}\\
  \caption{Map of the 301 precipitation stations with a good data coverage of the period 1981--2010 (blue dots), and the 30 stations with long archives (orange).}
	\label{fig:stations}
\end{figure}

\begin{figure}[t]
	\noindent\includegraphics[width=39pc,angle=0]{figures/boxplot-all-methods-values.pdf}\\
	\caption{CRPSS score for all stations and for all considered AMs and reanalysis datasets. A higher CRPSS score means better performance. The parameters of the AMs were calibrated for every station, every dataset, and every method.}
	\label{fig:comparison_values}
\end{figure}

\begin{figure}[t]
	\noindent\includegraphics[width=39pc,angle=0]{figures/boxplot-all-methods-relative.pdf}\\
	\caption{Effect of the reanalysis dataset on the performance isolated by processing the improvement (\%) in CRPSS for one dataset compared to the mean performance on all datasets, per station and per method.}
	\label{fig:comparison_relative}
\end{figure}

\begin{figure}[t]
	\noindent\includegraphics[width=39pc,angle=0]{figures/boxplot-all-methods-relative-eq0.pdf}\\
	\caption{Same as Fig. \ref{fig:comparison_relative}, but for target days with no precipitation. Please note the difference of the scales}
	\label{fig:comparison_relative_P0}
\end{figure}

\begin{figure}[t]
	\noindent\includegraphics[width=39pc,angle=0]{figures/boxplot-all-methods-relative-q99.pdf}\\
	\caption{Same as Fig. \ref{fig:comparison_relative}, but for target days with precipitation above the all days 99th percentile. Please note the difference of the scales}
	\label{fig:comparison_relative_Pq99}
\end{figure}

\begin{figure}[t]
	\noindent\includegraphics[width=38pc,angle=0]{figures/maps-best-methods.pdf}\\
	\caption{Best method per station for the different datasets. NR-2 and JRA-55C are not shown as they are similar to NR-1 and JRA-55 respectively.}
	\label{fig:map_best_methods}
\end{figure}

\begin{figure}[t]
	\noindent\includegraphics[width=19pc,angle=0]{figures/plot-impact-resolution.pdf}\\
	\caption{Impact (\% of CRPSS) of a decrease in grid resolution (degrees) for different datasets and AMs.}
	\label{fig:plot_impact_resolution}
\end{figure}

\begin{figure}[t]
	\noindent\includegraphics[width=19pc,angle=0]{figures/plot-impact-length.pdf}\\
	\caption{Impact (\% of CRPSS) of an increase in the archive length (years) for different datasets and AMs. Results for the 4Z method are displayed along with the 2Z method, but with dashed lines.}
	\label{fig:plot_impact_length}
\end{figure}



\end{document}